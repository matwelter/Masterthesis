%% Dokumentenklasse (Koma Script) -----------------------------------------
\documentclass[%
   %draft,     % Entwurfsstadium
   final,      % fertiges Dokument
   12pt,
   headings=big,       % gro�e �berschriften
   ngerman,           % wird an andere Pakete weitergereicht
   a4paper,
   BCOR5mm,          % Zusaetzlicher Rand auf der Innenseite
   DIV12,            % Seitengroesse (siehe Koma Skript Dokumentation !)
   %DIV=calc,
   1.1headlines,     % Zeilenanzahl der Kopfzeilen
   pagesize,         % Schreibt die Papiergroesse in die Datei.
   oneside,
%   twoside,          % Seitenraender f�r zweiseitiges Layout
   openright,        % Kapitel beginnen immer auf der rechten Seite
   titlepage,        % Titel als einzelne Seite ('titlepage' Umgebung) 
   parskip=false,        % Einger�ckt (Standard)
   headsepline,      % Linie unter Kolumnentitel
   chapterprefix=false,  % keine Ausgabe von 'Kapitel:'      
	 toc=bibliography, % Literaturverzeichnis ins Inhaltsverzeichnis
	 toc=graduated,		% eingereuckte Gliederung des Inhaltsverzeichnisses
	 toc=listof,			% Tabellen- und Abbildungsverzeichnis ins Inhaltsverzeichnis
   numbers=noenddot, % �erschriftnummerierung ohne Punkt, siehe DUDEN !
   %cleardoubleplain=plain,
	 footinclude=false,
   %fleqn,            % Formeln werden linksbuendig angezeigt
]{scrbook}%     Klassen: scrartcl, scrreprt, scrbook
% -------------------------------------------------------------------------


\usepackage[latin1]{inputenc}



%%% Preambel
%%% Frei nach einer Vorlage von Matthias Pospiech
%%% Modifiziert von Frederik Beer



\input{preambel/preambel-commands}
%
%%% Doc: www.cs.brown.edu/system/software/latex/doc/calc.pdf
\usepackage{calc}

%%% Doc: ftp://tug.ctan.org/pub/tex-archive/macros/latex/contrib/xcolor/xcolor.pdf
\usepackage[
	table % Load for using rowcolors command in tables
]{xcolor}


%%% Doc: http://www.ctan.org/tex-archive/macros/latex/contrib/listings/listings.pdf
\usepackage{listings}

%% Settings for the listing stuff
\definecolor{hellgrau}{rgb}{0.9,0.9,0.9}
\definecolor{colKeys}{rgb}{0,0,1}
\definecolor{colIdentifier}{rgb}{0,0,0}
\definecolor{colComments}{rgb}{1,0,0}
\definecolor{colString}{rgb}{0,0.5,0}

\lstset{%
   morekeywords={AND,ASC,avg,CHECK,COMMIT,count,DECODE,DESC,DISTINCT,%
                 GROUP,IN,LIKE,NUMBER,ROLLBACK,SUBSTR,sum,VARCHAR2}%
}
\lstset{%
    float=hbp,%
    basicstyle=\ttfamily\small, %
    %identifierstyle=\color{colIdentifier}, %
    %keywordstyle=\color{colKeys}, %
    %stringstyle=\color{colString}, %
    %commentstyle=\color{colComments}, %
    columns=flexible, %
    tabsize=2, %
    frame=single, %
    extendedchars=true, %
    showspaces=false, %
    showstringspaces=false, %
    numbers=left, %
    numberstyle=\tiny, %
    breaklines=true, %
    backgroundcolor=\color{hellgrau}, %
    breakautoindent=true, %
    captionpos=b%
}

%%% Doc: ftp://tug.ctan.org/pub/tex-archive/macros/latex/required/babel/babel.pdf
\usepackage[
%	german,
	ngerman,
%	english,
%	frensh,
]{babel}


%%% Doc: ftp://tug.ctan.org/pub/tex-archive/macros/latex/required/graphics/grfguide.pdf
% Will be loaded by pstools
%\usepackage[pdftex]{graphicx}

%%% Doc: http://mirrors.ctan.org/macros/latex/contrib/pstool/pstool.pdf
\usepackage{pstool}

%%% Doc: ftp://tug.ctan.org/pub/tex-archive/macros/latex/contrib/oberdiek/epstopdf.pdf
%%% Macht nur Sinn bei der Verwendung von PDFLatex und EPS Grafiken, wandelt diese dann
%%% automatisch in PDFs um.
\usepackage{epstopdf}


%%% Doc: ftp://tug.ctan.org/pub/tex-archive/macros/latex/required/amslatex/math/amsldoc.pdf
\usepackage[
   centertags, % (default) center tags vertically
   %tbtags,    % 'Top-or-bottom tags': For a split equation, place equation numbers level
               % with the last (resp. first) line, if numbers are on the right (resp. left).
   sumlimits,  %(default) Place the subscripts and superscripts of summation
               % symbols above and below
   %nosumlimits, % Always place the subscripts and superscripts of summation-type
               % symbols to the side, even in displayed equations.
   intlimits,  % Like sumlimits, but for integral symbols.
   %nointlimits, % (default) Opposite of intlimits.
   namelimits, % (default) Like sumlimits, but for certain 'operator names' such as
               % det, inf, lim, max, min, that traditionally have subscripts placed underneath
               % when they occur in a displayed equation.
   %nonamelimits, % Opposite of namelimits.
   %leqno,     % Place equation numbers on the left.
   %reqno,     % Place equation numbers on the right.
   %fleqn,     % Position equations at a fixed indent from the left margin rather than
               % centered in the text column.
]{amsmath} %
\usepackage{amssymb}

%% Doc: ftp://tug.ctan.org/pub/tex-archive/macros/latex/contrib/marginnote/marginnote.pdf
\usepackage{marginnote}

%% Doc: (inside relsize.sty )
%% ftp://tug.ctan.org/pub/tex-archive/macros/latex/contrib/misc/relsize.sty
\usepackage{relsize}

%% Doc: ftp://tug.ctan.org/pub/tex-archive/macros/latex/contrib/ms/ragged2e.pdf
\usepackage{ragged2e}


\usepackage[T1]{fontenc} % T1 Schrift Encoding
\usepackage{textcomp}	 % Zusatzliche Symbole (Text Companion font extension)

%%% Schriften werden in Fonts.tex geladen
\input{preambel/Fonts}

%%% Doc: ftp://tug.ctan.org/pub/tex-archive/macros/latex/contrib/mh/doc/mathtools.pdf
\usepackage[fixamsmath,disallowspaces]{mathtools}

%%% Doc: http://www.ctan.org/info?id=fixmath
\usepackage{fixmath}

%%% Doc: ftp://tug.ctan.org/pub/tex-archive/macros/latex/contrib/onlyamsmath/onlyamsmath.dvi
\usepackage[
	all,
	warning
]{onlyamsmath}

%%% Doc: http://www.tex.ac.uk/ctan/macros/latex/contrib/koma-script/tocstyle.pdf
%%% Kann verwendet werden, wenn im Inhaltsverzeichnis �berall oder nirgends Punkte gew�nscht sind.
%\usepackage{tocstyle}
%\usetocstyle{allwithdot}
%\usetocstyle{noonewithdot}
%\usetocstyle{KOMAlike}

%------------------------------------------------------

% -- Vektor fett darstellen -----------------
% \let\oldvec\vec
% \def\vec#1{{\boldsymbol{#1}}} %Fetter Vektor
% \newcommand{\ve}{\vec} %
% -------------------------------------------

%%% Doc: ftp://tug.ctan.org/pub/tex-archive/macros/latex/contrib/was/icomma.dtx
\usepackage{icomma}

%%% Tauschen von Epsilon und andere:
% \let\ORGvarrho=\varrho
% \let\varrho=\rho
% \let\rho=\ORGvarrho
%
\let\ORGvarepsilon=\varepsilon
\let\varepsilon=\epsilon
\let\epsilon=\ORGvarepsilon
%
% \let\ORGvartheta=\vartheta
% \let\vartheta=\theta
% \let\theta=\ORGvartheta
%
% \let\ORGvarphi=\varphi
% \let\varphi=\phi
% \let\phi=\ORGvarphi


%%% Doc: ftp://tug.ctan.org/pub/tex-archive/macros/latex/contrib/booktabs/booktabs.pdf
\usepackage{booktabs}

% Tabellen ueber mehere Seiten
% ----------------------------
%%% Doc: ftp://tug.ctan.org/pub/tex-archive/macros/latex/contrib/carlisle/ltxtable.pdf
% \usepackage{ltxtable} % Longtable + tabularx
                        % (multi-page tables) + (auto-sized columns in a fixed width table)
% -> nach hyperref laden
\LoadPackageLater{ltxtable}


%%% Doc: ftp://tug.ctan.org/pub/tex-archive/macros/latex/contrib/soul/soul.pdf
\usepackage{soul}		            % Unterstreichen, Sperren
%%% Doc: ftp://tug.ctan.org/pub/tex-archive/macros/latex/contrib/misc/url.sty
\usepackage{url} % Setzen von URLs. In Verbindung mit hyperref sind diese auch aktive Links.

%%%% Doc: ftp://tug.ctan.org/pub/tex-archive/macros/latex/contrib/footmisc/footmisc.pdf
\usepackage[
   bottom,      % Footnotes appear always on bottom. This is necessary
                % especially when floats are used
   stable,      % Make footnotes stable in section titles
   perpage,     % Reset on each page
   %para,       % Place footnotes side by side of in one paragraph.
   %side,       % Place footnotes in the margin
   ragged,      % Use RaggedRight
   %norule,     % suppress rule above footnotes
   multiple,    % rearrange multiple footnotes intelligent in the text.
   %symbol,     % use symbols instead of numbers
]{footmisc}

%% Einruecken der Fussnote einstellen
%\setlength\footnotemargin{10pt}

%--- footnote counter documentweit durchlaufend ------------------------------
%\usepackage{chngcntr}
%\counterwithout{footnote}{chapter}
%-----------------------------------------------------------------------------


%%% Doc: Documentation inside dtx File
\usepackage[ngerman]{varioref} % Intelligente Querverweise


%%% Doc: ftp://tug.ctan.org/pub/tex-archive/macros/latex/contrib/enumitem/enumitem.pdf
% Better than 'paralist' and 'enumerate' because it uses a keyvalue interface !
% Do not load together with enumerate.
\IfPackageNotLoaded{enumerate}{
	\usepackage{enumitem}
}

%% Doc: ftp://tug.ctan.org/pub/tex-archive/macros/latex/contrib/csquotes/csquotes.pdf
% Advanced features for clever quotations
\usepackage[%
   babel,            % the style of all quotation marks will be adapted
                     % to the document language as chosen by 'babel'
   german=quotes,		% Styles of quotes in each language
   english=british,
   french=guillemets
]{csquotes}

% All facilities which take a 'cite' argument will not insert
% it directly. They pass it to an auxiliary command called \mkcitation
% which  may be redefined to format the citation.
\renewcommand*{\mkcitation}[1]{{\,}#1}
\renewcommand*{\mkccitation}[1]{ #1}

\SetBlockThreshold{2} % Anzahl von Zeilen

\newenvironment{myquote}%
	{\begin{quote}\small}%
	{\end{quote}}%
\SetBlockEnvironment{myquote}
%\SetCiteCommand{} % Changes citation command


%%% Doc: ftp://tug.ctan.org/pub/tex-archive/macros/latex/contrib/natbib/natbib.pdf
\usepackage[%
	%round,	%(default) for round parentheses;
	square,	% for square brackets;
	%curly,	% for curly braces;
	%angle,	% for angle brackets;
	%colon,	% (default) to separate multiple citations with colons;
	comma,	% to use commas as separaters;
	%authoryear,% (default) for author-year citations;
	numbers,	% for numerical citations;
	%super,	% for superscripted numerical citations, as in Nature;
	sort,		% orders multiple citations into the sequence in which they appear in the list of references;
	sort&compress,    % as sort but in addition multiple numerical citations
                   % are compressed if possible (as 3-6, 15);
	%longnamesfirst,  % makes the first citation of any reference the equivalent of
                   % the starred variant (full author list) and subsequent citations
                   %normal (abbreviated list);
	%sectionbib,      % redefines \thebibliography to issue \section* instead of \chapter*;
                   % valid only for classes with a \chapter command;
                   % to be used with the chapterbib package;
	%nonamebreak,     % keeps all the authors names in a citation on one line;
                   %causes overfull hboxes but helps with some hyperref problems.
]{natbib}

%%% Bibliography styles according to DIN
%%% get from: http://www.ctan.org/tex-archive/biblio/bibtex/contrib/german/din1505/
%\bibliographystyle{alphadin}
%\bibliographystyle{alpha}
%\bibliographystyle{abbrvdin}
\bibliographystyle{bib/bst/plaindin}
%\bibliographystyle{unsrtdin}
%\bibliographystyle{bib/bst/alphadin-mod} % Modifiziert: Kleinere Abstaende vor ";" und kein "+" bei etal.

%%% Bibliography styles created with custombib
%%% Doc: ftp://tug.ctan.org/pub/tex-archive/macros/latex/contrib/custom-bib/makebst.pdf
%\bibliographystyle{bib/bst/AlphaDINFirstName}
%\bibliographystyle{bib/bst/alphadin}


% weitere BibTeX styles: http://www.cs.stir.ac.uk/~kjt/software/latex/showbst.html

%%% Doc: ftp://tug.ctan.org/pub/tex-archive/macros/latex/contrib/microtype/microtype.pdf
\ifpdf
\usepackage[%
	expansion=true, % better typography, but with much larger PDF file.
	protrusion=true
]{microtype}
\fi

%%% Doc: ftp://tug.ctan.org/pub/tex-archive/macros/latex/contrib/hyperref/doc/manual.pdf

\usepackage[
	  % Farben fuer die Links
    colorlinks=false,         % Links erhalten Farben statt Kaeten
    urlcolor=pdfurlcolor,    % \href{...}{...} external (URL)
    filecolor=pdffilecolor,  % \href{...} local file
    linkcolor=pdflinkcolor,  %\ref{...} and \pageref{...}
    % Links
    raiselinks=true,			 % calculate real height of the link
    %breaklinks,              % Links berstehen Zeilenumbruch, geht nicht bei DVI->PS
    backref=page,            % Backlinks im Literaturverzeichnis (section, slide, page, none)
    pagebackref=true,        % Backlinks im Literaturverzeichnis mit Seitenangabe
    verbose,
    hyperindex=true,         % backlinkex index
    linktocpage=true,        % Inhaltsverzeichnis verlinkt Seiten
    hyperfootnotes=false,     % Keine Links auf Fussnoten
    % Bookmarks
    bookmarks=true,          % Erzeugung von Bookmarks fuer PDF-Viewer
    bookmarksopenlevel=1,    % Gliederungstiefe der Bookmarks
    bookmarksopen=true,      % Expandierte Untermenues in Bookmarks
    bookmarksnumbered=false,  % Nummerierung der Bookmarks
    %bookmarkstype=toc,       % Art der Verzeichnisses
    % Anchors
    plainpages=false,        % Anchors even on plain pages ?
    pageanchor=true,         % Pages are linkable
    % PDF Informationen
    pdftitle={},             % Titel
    pdfauthor={Autor},            % Autor
    pdfcreator={LaTeX, hyperref, KOMA-Script}, % Ersteller
    %pdfproducer={pdfeTeX 1.10b-2.1} %Produzent
    pdfstartview=FitH,       % Dokument wird Fit Width geaefnet
    pdfpagemode=UseOutlines, % Bookmarks im Viewer anzeigen
    %pdfpagelabels=true,      % set PDF page labels
		%dvipdfm,								 % Links auch wenn PDF �ber DVI Umweg erstellt wird
		pdfborder={0 0 0},
 ]{hyperref}


\IfPackageLoaded{backref}{
   % % Change Layout of Backref
   \renewcommand*{\backref}[1]{%
   	% default interface
   	% #1: backref list
   	%
   	% We want to use the alternative interface,
   	% therefore the definition is empty here.
   }%
   \renewcommand*{\backrefalt}[4]{%
   	% alternative interface
   	% #1: number of distinct back references
   	% #2: backref list with distinct entries
   	% #3: number of back references including duplicates
   	% #4: backref list including duplicates
   	\mbox{(Zitiert auf %
   	\ifnum#1=1 %
		   Seite~%
	   \else
   		Seiten~%
   	\fi
   	#2)}%
   }
}

%%% Doc: ftp://tug.ctan.org/pub/tex-archive/macros/latex/contrib/oberdiek/hypcap.pdf
% Links auf Gleitumgebungen springen nicht zur Beschriftung,
% sondern zum Anfang der Gleitumgebung
\IfPackageLoaded{hyperref}{%
	\usepackage[figure]{hypcap}
}

% Auch Abbildung und nicht nur die Nummer wird zum Link (abgeleitet
% aus Posting von Heiko Oberdiek (d09n5p$9md$1@news.BelWue.DE);
% Verwendung: In \abbvref{label} ist ein Beispiel dargestellt
\providecommand*{\abbvrefname}{Abbildung}
\newcommand*{\abbvref}[1]{%
  \hyperref[#1]{\abbvrefname}\vref{#1}%
}

%%% Doc: ftp://tug.ctan.org/pub/tex-archive/macros/latex/contrib/pdfpages/pdfpages.pdf
\usepackage{pdfpages} % Include pages from external PDF documents in LaTeX documents

% Pakete Laden die nach Hyperref geladen werden sollen
\LoadPackagesNow % (ltxtable, tabularx)


%%% Doc: only dtx Package
\usepackage{float}             % Stellt die Option [H] fuer Floats zur Verfgung

%%% Doc: No Documentation
\usepackage{flafter}          % Floats immer erst nach der Referenz setzen

% Defines a \FloatBarrier command, beyond which floats may not
% pass; useful, for example, to ensure all floats for a section
% appear before the next \section command.
%\usepackage[
%	section		% "\section" command will be redefined with "\FloatBarrier"
%]{placeins}


%%% Doc: ftp://tug.ctan.org/pub/tex-archive/macros/latex/contrib/subfig/subfig.pdf
\usepackage{subfig} % Layout wird weiter unten festgelegt !

%%% Doc: ftp://tug.ctan.org/pub/tex-archive/macros/latex/contrib/wrapfig/wrapfig.sty
\usepackage{wrapfig}	        % defines wrapfigure and wrapfloat
%\setlength{\wrapoverhang}{\marginparwidth} % aeerlapp des Bildes ...
%\addtolength{\wrapoverhang}{\marginparsep} % ... in den margin
\setlength{\intextsep}{0.75\baselineskip} % Platz ober- und unterhalb des Bildes
% \intextsep ignoiert bei draft ???
%\setlength{\columnsep}{1em} % Abstand zum Text


% Make float placement easier
\renewcommand{\floatpagefraction}{.75} % vorher: .5
\renewcommand{\textfraction}{.1}       % vorher: .2
\renewcommand{\topfraction}{.8}        % vorher: .7
\renewcommand{\bottomfraction}{.5}     % vorher: .3
\setcounter{topnumber}{3}              % vorher: 2
\setcounter{bottomnumber}{2}           % vorher: 1
\setcounter{totalnumber}{5}            % vorher: 3

%%% Doc: http://tug.ctan.org/tex-archive/macros/latex/contrib/auto-pst-pdf/auto-pst-pdf.pdf
%\usepackage[
%latex={-interaction=nonstopmode},
%crop=off,runs=2
%]{auto-pst-pdf} %use [off] to stop compilation

%%% Doc: ftp://tug.ctan.org/pub/tex-archive/macros/latex/contrib/psfrag/pfgguide.pdf
%\usepackage{psfrag}	% Ersetzen von Zeichen in eps Bildern

%%% Doc: http://www.ctan.org/tex-archive/macros/latex/contrib/sidecap/sidecap.pdf
\usepackage[%
%	outercaption,%	(default) caption is placed always on the outside side
%	innercaption,% caption placed on the inner side
%	leftcaption,%  caption placed on the left side
	rightcaption,% caption placed on the right side
%	wide,%			caption of float my extend into the margin if necessary
%	margincaption,% caption set into margin
	ragged,% caption is set ragged
]{sidecap}

\renewcommand\sidecaptionsep{2em}
%\renewcommand\sidecaptionrelwidth{20}
\sidecaptionvpos{table}{c}
\sidecaptionvpos{figure}{c}

%%% Seems to be needed for glossaries on newer miktex installations
\usepackage{datatool}
%%% Doc: http://mirror.informatik.uni-mannheim.de/pub/mirrors/tex-archive/macros/latex/contrib/glossaries/glossaries-manual.html
\usepackage[ngerman]{translator}
%Paket laden
\usepackage[
nonumberlist, %keine Seitenzahlen anzeigen
acronym,      %ein Abk�rzungsverzeichnis erstellen
toc,          %Eintr�ge im Inhaltsverzeichnis
section]      %im Inhaltsverzeichnis auf section-Ebene erscheinen
{glossaries}

%Ein eigenes Symbolverzeichnis erstellen
\newglossary[slg]{symbolslist}{syi}{syg}{Symbolverzeichnis}

%Den Punkt am Ende jeder Beschreibung deaktivieren
\renewcommand*{\glspostdescription}{}

%Glossar-Befehle anschalten
\makeglossaries


%%% Doc: http://tug.ctan.org/tex-archive/macros/latex/contrib/siunitx/siunitx.pdf
\usepackage{siunitx}
%Normale im LaTeX-Dokument verwendete Schriftart nutzen
\sisetup{detect-all}

%%% Doc: http://ftp.gwdg.de/pub/ctan/macros/latex/contrib/ellipsis/ellipsis.pdf
\usepackage{ellipsis}  % >>Intelligente<< \dots


%%% Doc: ftp://tug.ctan.org/pub/tex-archive/macros/latex/contrib/setspace/setspace.sty
\usepackage{setspace}
%\doublespace	        % 2-facher Abstand
\onehalfspace        % 1,5-facher Abstand



% BCOR
%    current  % Satzspiegelberechnung mit dem aktuell gültigen BCOR-Wert erneut
%             % durchführen.
% DIV
%    calc     % Satzspiegelberechnung einschließlich Ermittlung eines guten
%             % DIV-Wertes erneut durchführen.
%    classic  % Satzspiegelberechnung nach dem
%             % mittelalterlichen Buchseitenkanon
%             % (Kreisberechnung) erneut durchführen.
%    current  % Satzspiegelberechnung mit dem aktuell gültigen DIV-Wert erneut
%             % durchführen.
%    default  % Satzspiegelberechnung mit dem Standardwert für das aktuelle
%             % Seitenformat und die aktuelle Schriftgröße erneut durchführen.
%             % Falls kein Standardwert existiert calc anwenden.
%    last     % Satzspiegelberechnung mit demselben DIV -Argument, das beim
%             % letzten Aufruf angegeben wurde, erneut durchführen

%\raggedbottom     % Variable Seitenhoehen zulassen

% Farben ================================================================

\IfDefined{definecolor}{%

% Farbe der Ueberschriften
%\definecolor{sectioncolor}{RGB}{0, 51, 153} % Blau
%\definecolor{sectioncolor}{RGB}{0, 25, 152}    % Blau (dunkler))
\definecolor{sectioncolor}{RGB}{0, 0, 0}    % Schwarz
%
% Farbe des Textes
\definecolor{textcolor}{RGB}{0, 0, 0}        % Schwarz
%
% Farbe fuer grau hinterlegte Boxen (fuer Paket framed.sty)
\definecolor{shadecolor}{gray}{0.90}

% Farben fuer die Links im PDF
\definecolor{pdfurlcolor}{rgb}{0.6,0,0}
\definecolor{pdffilecolor}{rgb}{0,0.5,0}
\definecolor{pdflinkcolor}{rgb}{0,0,0.75}

% Farben fuer Listings
\colorlet{stringcolor}{green!40!black!100}
\colorlet{commencolor}{blue!0!black!100}

} % Endif

%% Aussehen der URLS======================================================

%fuer URL (nur wenn url geladen ist)
\IfDefined{urlstyle}{
	\urlstyle{tt} %sf
}

%% Kopf und Fusszeilen====================================================
%%% Doc: ftp://tug.ctan.org/pub/tex-archive/macros/latex/contrib/koma-script/scrguide.pdf
\usepackage[%
   automark,         % automatische Aktualisierung der Kolumnentitel
   nouppercase,      % Grossbuchstaben verhindern
   %markuppercase    % Grossbuchstaben erzwingen
   %markusedcase     % vordefinierten Stil beibehalten
   %komastyle,       % Stil von Koma Script
   %standardstyle,   % Stil der Standardklassen
]{scrpage2}

\IfElseChapterDefined{%
   \pagestyle{scrheadings} % Seite mit Headern
}{
   \pagestyle{scrplain} % Seiten ohne Header
}
%\pagestyle{empty} % Seiten ohne Header
\clearscrheadings
%\clearscrplain
%
% Was steht wo...
\IfElseChapterDefined{
   % Oben aussen: Kapitel und Section
   % Unten aussen: Seitenzahl
   % \ohead{\headmark} % Oben außen: Setzt Kapitel und Section automatisch
   % \ofoot[\pagemark]{\pagemark}
   % oder...
   % Oben aussen: Seitenzahlen
   % Oben innen: Kapitel und Section
   \cfoot{\pagemark}
   \ohead{\headmark}
}{
   \cfoot[\pagemark]{\pagemark} % Mitte unten: Seitenzahlen bei plain
}
% Vollstaendige Liste der moeglichen Positionierungen
% \lehead[scrplain-links-gerade]{scrheadings-links-gerade}
% \cehead[scrplain-mittig-gerade]{scrheadings-mittig-gerade}
% \rehead[scrplain-rechts-gerade]{scrheadings-rechts-gerade}
% \lefoot[scrplain-links-gerade]{scrheadings-links-gerade}
% \cefoot[scrplain-mittig-gerade]{scrheadings-mittig-gerade}
% \refoot[scrplain-rechts-gerade]{scrheadings-rechts-gerade}
% \lohead[scrplain-links-ungerade]{scrheadings-links-ungerade}
% \cohead[scrplain-mittig-ungerade]{scrheadings-mittig-ungerade}
% \rohead[scrplain-rechts-ungerade]{scrheadings-rechts-ungerade}
% \lofoot[scrplain-links-ungerade]{scrheadings-links-ungerade}
% \cofoot[scrplain-mittig-ungerade]{scrheadings-mittig-ungerade}
% \rofoot[scrplain-rechts-ungerade]{scrheadings-rechts-ungerade}
% \ihead[scrplain-innen]{scrheadings-innen}
% \chead[scrplain-zentriert]{scrheadings-zentriert}
% \ohead[scrplain-außen]{scrheadings-außen}
% \ifoot[scrplain-innen]{scrheadings-innen}
% \cfoot[scrplain-zentriert]{scrheadings-zentriert}
% \ofoot[scrplain-außen]{scrheadings-außen}


%\usepackage{lastpage} % Stellt 'LastPage' zur Verfuegung
%\cfoot[Seite \pagemark~von \pageref{LastPage}]{} % Seitenzahl von Anzahl Seiten

% Angezeigte Abschnitte im Header
\IfElseChapterDefined{
   \automark[section]{chapter} %[rechts]{links}
}{
   \automark[subsection]{section} %[rechts]{links}
}
%
% Linien (moegliche Kombination mit Breiten)
\IfChapterDefined{
   %\setheadtopline{}     % modifiziert die Parameter fuer die Linie ueber dem Seitenkopf
   \setheadsepline{.4pt}[\color{black}]
                         % modifiziert die Parameter fuer die Linie zwischen Kopf
                         % und Textkörper
   %\setfootsepline{}    % modifiziert die Parameter fuer die Linie zwischen Text
                         % und Fuß
   %\setfootbotline{}    % modifiziert die Parameter fuer die Linie unter dem Seitenfuss
}

% Groesse des Headers
\setlength{\headheight}{1.1\baselineskip}
% -> eingestellt �ber Option 'headlines'.

% Breite von Kopf und Fusszeile einstellen
% \setheadwidth[Verschiebung]{Breite}
% \setfootwidth[Verschiebung]{Breite}
% m�gliche Werte
% paper - die Breite des Papiers
% page - die Breite der Seite
% text - die Breite des Textbereichs
% textwithmarginpar - die Breite des Textbereichs inklusive dem Seitenrand
% head - die aktuelle Breite des Seitenkopfes
% foot - die aktuelle Breite des Seitenfusses
\setheadwidth[0pt]{text}
\setfootwidth[0pt]{text}


%% Fussnoten =============================================================
% Keine hochgestellten Ziffern in der Fussnote (KOMA-Script-spezifisch):
\deffootnote{1.5em}{1em}{\makebox[1.5em][l]{\thefootnotemark}}
\addtolength{\skip\footins}{\baselineskip} % Abstand Text <-> Fussnote

\setlength{\dimen\footins}{10\baselineskip} % Beschraenkt den Platz von Fussnoten auf 10 Zeilen

\interfootnotelinepenalty=10000 % Verhindert das Fortsetzen von
                                % Fussnoten auf der gegenüberligenden Seite


%% Schriften (Sections )==================================================

\IfElsePackageLoaded{fourier}{
   \newcommand\SectionFontStyle{\rmfamily}
}{
   \newcommand\SectionFontStyle{\sffamily}
}

% -- Koma Schriften --
\IfChapterDefined{%
   \setkomafont{chapter}{\huge\SectionFontStyle}    % Chapter
}
\setkomafont{sectioning}{\SectionFontStyle} %  % Titelzeilen % \bfseries
\setkomafont{pagenumber}{\small\SectionFontStyle}             % Seitenzahl
\setkomafont{pageheadfoot}{\small\sffamily}        % Kopfzeile
%\setkomafont{pagefoot}{\small\sffamily}        % Kopfzeile
\setkomafont{descriptionlabel}{\itshape}        % Kopfzeile
%

\addtokomafont{sectioning}{\color{sectioncolor}} % Farbe der Ueberschriften
\IfChapterDefined{%
	\addtokomafont{chapter}{\color{sectioncolor}} % Farbe der Ueberschriften
}
\renewcommand*{\raggedsection}{\raggedright} % Titelzeile linksbuendig, haengend
%
%% UeberSchriften (Chapter und Sections) =================================
% -- Ueberschriften komlett Umdefinieren --
%%% Doc: ftp://tug.ctan.org/pub/tex-archive/macros/latex/contrib/titlesec/titlesec.pdf
\usepackage{titlesec}

% -- Section Aussehen veraendern --
% --------------------------------
%% -> Section mit Unterstrich
% \titleformat{\section}
%   [hang]%[frame]display
%   {\usekomafont{sectioning}\Large}
%  {\thesection}
%   {6pt}
%   {}
%   [\titlerule \vspace{0.5\baselineskip}]
% --------------------------------

% -- Chapter Aussehen veraendern --
% --------------------------------
%--> Box mit (Kapitel + Nummer ) +  Name
% \titleformat{\chapter}[display]     % {command}[shape]
%   {\usekomafont{chapter}\filcenter} % format
%   {                                 % label
%   {\fcolorbox{black}{shadecolor}{
%   {\huge\chaptertitlename\mbox{\hspace{1mm}}\thechapter}
%   }}}
%   {1pc}                             % sep (from chapternumber)
%   {\vspace{1pc}}                    % {before}[after] (before chaptertitle and after)
% --------------------------------
%--> Kapitel + Nummer + Trennlinie + Name + Trennlinie
\titleformat{\chapter}[display]	% {command}[shape]
  {\usekomafont{chapter}\Large \color{black}}	% format
  {   										% label
  \LARGE\MakeUppercase{\chaptertitlename} \Huge \thechapter \filright%
  }%}
  {1pt}										% sep (from chapternumber)
  {\titlerule \vspace{0.9pc} \filright \color{sectioncolor}}   % {before}[after] (before chaptertitle and after)
  [\color{black} \vspace{0.9pc} \filright {\titlerule}]


%% Captions (Schrift, Aussehen) ==========================================

% % Folgende Befehle werden durch das Paket caption und subfig ersetzt !
% \setcapindent{1em} % Einrueckung der Beschriftung
% \setkomafont{caption}{\color{black}\small\sffamily\RaggedRight}  % Schrift fuer Caption
% \setkomafont{captionlabel}{\color{black}\small}   % Schrift fuer 'Abbildung' usw.

%%% Doc: ftp://tug.ctan.org/pub/tex-archive/macros/latex/contrib/caption/caption.pdf
\usepackage{caption}
% Aussehen der Captions
\captionsetup{
   margin = 10pt,
   font = {small,sf},
   labelfont = {small,bf},
   format = plain, % oder 'hang'
   indention = 0em,  % Einruecken der Beschriftung
   labelsep = colon, %period, space, quad, newline
   justification = RaggedRight, % justified, centering
   singlelinecheck = true, % false (true=bei einer Zeile immer zentrieren)
   position = bottom %top
}
%%% Bugfix Workaround
\DeclareCaptionOption{parskip}[]{}
\DeclareCaptionOption{parindent}[]{}

% Aussehen der Captions fuer subfigures (subfig-Paket)
\IfPackageLoaded{subfig}{
 \captionsetup[subfloat]{%
   margin = 10pt,
   font = {small,sf},
   labelfont = {small,bf},
   format = plain, % oder 'hang'
   indention = 0em,  % Einruecken der Beschriftung
   labelsep = space, %period, space, quad, newline
   justification = RaggedRight, % justified, centering
   singlelinecheck = true, % false (true=bei einer Zeile immer zentrieren)
   position = bottom, %top
   labelformat = parens % simple, empty % Wie die Bezeichnung gesetzt wird
 }
}

% Aendern der Bezeichnung fuer Abbildung und Tabelle
% \addto\captionsngerman{% "captionsgerman" fuer alte  Rechschreibung
%   \renewcommand{\figurename}{Abb.}%
%   \renewcommand{\tablename}{Tab.}%
% }

% Caption fuer nicht fliessende Umgebungen
%%% Doc: ftp://tug.ctan.org/pub/tex-archive/macros/latex/contrib/misc/capt-of.sty
\IfPackageNotLoaded{caption}{
	\usepackage{capt-of} % only load when caption is not loaded. Otherwise compiling will fail.
	%Usage: \captionof{table}[short Titel]{long Titel}
}
%


%%% Doc: ftp://tug.ctan.org/pub/tex-archive/macros/latex/contrib/mcaption/mcaption.pdf
% Captions in Margins
% \usepackage[
% 	top,
% 	bottom
% ]{mcaption}

%%% Example:
% \begin{figure}
%   \begin{margincap}[short caption]{margin caption}
%     \centering
%     \includegraphics{picture}
%   \end{margincap}
% \end{figure}



% \numberwithin{figure}{chapter} %Befehl zum Kapitelweise Nummerieren der Bilder, setzt `amsmath' vorraus
% \numberwithin{table}{chapter}  %Befehl zum Kapitelweise Nummerieren der Tabellen, setzt `amsmath' vorraus

%% Inhaltsverzeichnis (Schrift, Aussehen) sowie weitere Verzeichnisse ====

\setcounter{secnumdepth}{4}    % Abbildungsnummerierung mit groesserer Tiefe
\setcounter{tocdepth}{2}		 % Inhaltsverzeichnis mit groesserer Tiefe
%

% Inhalte von List of Figures
\IfPackageLoaded{subfig}{
	\setcounter{lofdepth}{1}  %1 = nur figures, 2 = figures + subfigures
}


% Auszufuehrende Befehle  ------------------------------------------------
\IfDefined{makeindex}{\makeindex}
\IfDefined{makenomenclature}{\makenomenclature}
\IfPackageLoaded{minitoc}{\ifundefined{chapter}{\dosecttoc}{\dominitoc}}


\listfiles
%------------------------------------------------------------------------

%DIF PREAMBLE EXTENSION ADDED BY LATEXDIFF
%DIF UNDERLINE PREAMBLE %DIF PREAMBLE
%\RequirePackage[normalem]{ulem} %DIF PREAMBLE
\usepackage[normalem]{ulem}
%\RequirePackage{color}\definecolor{RED}{rgb}{1,0,0}\definecolor{BLUE}{rgb}{0,0,1} %DIF PREAMBLE
\providecommand{\DIFadd}[1]{{\protect\color{blue}\uwave{#1}}} %DIF PREAMBLE
\providecommand{\DIFdel}[1]{{\protect\color{red}\sout{#1}}}                      %DIF PREAMBLE
%DIF SAFE PREAMBLE %DIF PREAMBLE
\providecommand{\DIFaddbegin}{} %DIF PREAMBLE
\providecommand{\DIFaddend}{} %DIF PREAMBLE
\providecommand{\DIFdelbegin}{} %DIF PREAMBLE
\providecommand{\DIFdelend}{} %DIF PREAMBLE
%DIF FLOATSAFE PREAMBLE %DIF PREAMBLE
\providecommand{\DIFaddFL}[1]{\DIFadd{#1}} %DIF PREAMBLE
\providecommand{\DIFdelFL}[1]{\DIFdel{#1}} %DIF PREAMBLE
\providecommand{\DIFaddbeginFL}{} %DIF PREAMBLE
\providecommand{\DIFaddendFL}{} %DIF PREAMBLE
\providecommand{\DIFdelbeginFL}{} %DIF PREAMBLE
\providecommand{\DIFdelendFL}{} %DIF PREAMBLE
%DIF END PREAMBLE EXTENSION ADDED BY LATEXDIFF

%%% Neue Befehle
% % redefine \textmu to other mu commands usefull inside text
% \renewcommand{\textmu}{$\upmu$}

\newcommand{\engl}[1]{\textit{#1}}
\newcommand{\aclui}[1]{\textit{\aclu{#1}}}
\newcommand{\acli}[1]{\textit{\acl{#1}}}

\input{macros/TableCommands}

%%% Silbentrennung
\hyphenation{di-vi-sion}
\hyphenation{RTCM}
\hyphenation{NTRIP}
\hyphenation{DGPS}
\hyphenation{Kor-rek-tur-daten}
\hyphenation{ProviderID}
\hyphenation{ESG-Ac-cess-Descriptor}
\hyphenation{ESG-Pro-vi-der-Dis-covery-Descriptor}
\hyphenation{DVB}
\hyphenation{Comm-API}
\hyphenation{Po-si-tion}
\hyphenation{FLUTE}
\hyphenation{AGPS}
\hyphenation{NDGPS}
\hyphenation{WADGPS}
\hyphenation{LADGPS}
\hyphenation{Fahr-spur-as-sis-tent}
\hyphenation{GLONASS}
\hyphenation{ger�t-internen}
\hyphenation{Dif-ferential-glei-chungen}

%%% Abk�rzungen, Glossar und Symbolverzeichnis
%%%
%Symbole
%%%
\newglossaryentry{symb:Pi}{
name=$\pi$,
description={Die Kreiszahl.},
sort=symbolpi, type=symbolslist
}
\newglossaryentry{symb:Phi}{
name=$\varphi$,
description={Ein beliebiger Winkel.},
sort=symbolphi, type=symbolslist
}
\newglossaryentry{symb:Lambda}{
name=$\lambda$,
description={Eine beliebige Zahl, mit der der nachfolgende Ausdruck
multipliziert wird.},
sort=symbollambda, type=symbolslist
}

%%%
%Abk�rzungen
%%%
\newacronym{fcu}{FCU}{Flight Control Unit}
\newacronym{imu}{IMU}{Inertial Measurement Unit}
\newacronym{llp}{LLP}{Low Level Processor}
\newacronym{hlp}{HLP}{High Level Processor}
\newacronym{uav}{UAV}{Unmanned Aerial Vehicle}
\newacronym{ros}{ROS}{Robot Operation System}
\newacronym{asctec}{AscTec}{Ascending Technologies}
\newacronym{enu}{ENU}{East-North-Up}
\newacronym{ned}{NED}{North-East-Down}
\newacronym{icp}{ICP}{Interative Closest Point}
\newacronym{iir}{IIR}{Infinite Impulse Response}
\newacronym{fir}{FIR}{Finite Impulse Response}
\newacronym{foaw}{FOAW}{First-Order Adaptive Windowing}
\newacronym{pc}{PC}{Personal Computer}
\newacronym{rfid}{RFID}{Radio-Frequency IDentification}
\newacronym{wlan}{WLAN}{Wireless Local Area Network}

\newacronym{mems}{MEMS}{Microelectromechanical systems}

%Eine Abk�rzung mit Glossareintrag
\newacronym{spi}{SPI}{Serial Peripheral Interface \protect\glsadd{glos:spi}}
\newacronym{uart}{UART}{Universal Asynchronous Receiver/Transmitter\protect\glsadd{glos:uart}}
\newacronym{i2c}{I$^2$C}{Inter Integrated Circuit\protect\glsadd{glos:i2c}}
\newacronym{ip}{IP}{Internet Protocol\protect\glsadd{glos:ip}}
\newacronym{vslam}{VSLAM}{Visual Simultaneous Localization and Mapping \glsadd{gls:vslam}}
\newacronym{ftdi}{FTDI}{Future Technology Devices International \glsadd{gls:ftdi}}
\newacronym{ssh}{SSH}{Secure Shell \glsadd{gls:ssh}}
\newacronym{slam}{SLAM}{Simultaneous Localization and Mapping \glsadd{gls:slam}}
%%%
%Glossareintr�ge
%%%
\newglossaryentry{glos:spi}{
	name=Serial Peripherial Interface,
	description={Ein von Motorola entwickelter Standard eines synchronen seriellen Datenbusses zur Vernetzung zweier digitaler Schaltungen nach dem Master-Slave-Prinzip.}}
\newglossaryentry{glos:uart}{
	name= Asynchronous Receiver/Transmitter,
	description={Eine g�ngige serielle Schnittstelle zum Senden und Empfangen von Daten �ber eine Datenleitung. Bildet den Standard der seriellen Schnittstellen an \gls{pc}s und Mikrocontrollern.}}
\newglossaryentry{glos:i2c}{
	name= Inter Integrated Circuit,
	description={Von Philips Semiconductor entwickelter serielles Bussystem. Haupts�chlich zur ger�tinternen Kommunikation zwischen verschieden Schaltungsteilen eingesetzt.}}
\newglossaryentry{glos:ip}{
	name= Internet Protocol,
	description={Ist das am weitesten verbreitete Netzwerkprotokoll in Computernetzen und stellt die Grundlage des Internets dar.}}
\newglossaryentry{gls:slam}{
	name= Simultaneous Localization and Mapping,
	description={SLAM ist eine Methode der mobilen Robotik zur Sch�tzung der eigenen Position in einer unbekannten Umgebung. Daf�r wird eine Karte des Raumes aus den Messdaten erstellt. Im deutschen steht SLAM f�r Simultane Lokalisierung und Kartenerstellung.}}
\newglossaryentry{gls:ftdi}{
	name= Future Technology Devices International,
	description={Die Firma ist f�r ihre USB-RS232-Interface-Chips bekannt. Diese erm�glichen es eine serielle Schnittstelle �ber USB verf�gbar zu machen.}}
\newglossaryentry{gls:ssh}{
	name= Secure Shell,
	description={Netzwerkprotokoll f�r verschl�sselte Verbindungen.}}
\newglossaryentry{gls:vslam}{
	name= Visual Simultaneous Localization and Mapping,
	description={Der Zusatz V sagt dabei aus, dass diese Messdaten von einem visuellen Sensor, sprich einer Kamera zur Verf�gung gestellt werden.}}






%%% Alles serifenlos! (au�er mathe)
\renewcommand{\familydefault}{\sfdefault}
%\usepackage{helvet}
\usepackage[osf]{mathpazo}
%\usepackage{hvmaths}

%Acronymfonts umdefinieren (nur f�r Paket 'Acronym' interessant)
%Ausgeschriebenes Kursiv, Abkuerzung und Klammern normal
%\renewcommand*{\acffont}[1]{\textit{#1}}
%\renewcommand*{\acfsfont}[1]{\textnormal{#1}}

%Abk�rzungen werden kursiv gestellt (Paket 'glossaries')
\renewcommand*{\glstextformat}[1]{\textit{#1}} 

%Hurenkinder und Schusterjungen verhindern
\clubpenalty = 10000
\widowpenalty = 10000 
\displaywidowpenalty = 10000

%Trennen von Inline Formeln unterbinden
\relpenalty=9999
\binoppenalty=9999


%% Dokument Beginn %%%%%%%%%%%%%%%%%%%%%%%%%%%%%%%%%%%%%%%%%%%%%%%%%%%%%%%%

\begin{document}

% Deckblatt
\pagenumbering{gobble}
\begin{titlepage}
\label{sec:Titel}
\pdfbookmark[0]{Titelseite}{sec:Titel}

\setcounter{page}{0}

\begin{center}
\large\textbf{Friedrich-Alexander-Universit�t Erlangen-N�rnberg}
\end{center}
\vspace{0.2cm}
\centering
\includegraphics[width=6.5cm]{images/FAU_tech_logo}
\vspace{0.2cm}
~\\\large\textbf{Lehrstuhl f�r Informationstechnik\\mit dem Schwerpunkt Kommunikationselektronik} 
\vspace{0.2cm}
\begin{center}
\includegraphics[width=5cm]{images/logo_like}
\end{center}

\begin{center}
Professor Dr.-Ing. J�rn Thielecke	
\end{center}
\vspace{0.8cm}
\begin{center}
	\large\textbf{Diplomarbeit}\\
	~\\
	\textbf{Thema:}\\
\end{center}
\begin{center}
Horizontale Geschwindigkeitsregelung eines Quadrocopter mit Hilfe von Laserdaten
\end{center}
\vspace{1.5cm}
\begin{flushleft}

		\begin{tabular}{ll}
			Bearbeiter: 	& B.Eng Matthias Welter	 							\\
			~\\
			Betreuer: 		& Dipl.-Inf. Manuel Stahl\\
							& Dipl.-Ing. Christian Strobel\\
						    
			~\\
			Beginn: 		  & 	01. August 2014 							\\
			Ende: 		      & 	31. Januar 2015 						\\
		\end{tabular}
		
\end{flushleft}

\end{titlepage}
\chapter*{Best\"atigung}\thispagestyle{empty}
\label{sec:Bestaetigung}
\pdfbookmark[0]{Best�tigung}{sec:Bestaetigung}

\vspace*{2cm}
Erkl�rung:
~\\
Ich versichere, dass ich die Arbeit ohne fremde Hilfe und ohne Benutzung anderer als der angegebenen Quellen angefertigt habe und, dass die Arbeit in gleicher oder �hnlicher Form noch keiner anderen Pr�fungsbeh�rde vorgelegen hat und von dieser als Teil einer Pr�fungsleistung angenommen wurde. Alle Ausf�hrungen, die w�rtlich oder sinngem�� �bernommen wurden, sind als solche gekennzeichnet.


\vspace*{2cm}

~\\
\begin{tabular}{lc}
	Erlangen, den 31.01.2015	&	\underline{ \ \ \ \ \ \ \ \ \ \ \ \ \ \ \ \ \ \ \ \ \ \ \ \ \ \ \ \ \ \ \ \ \ \ }\\
												    & \small{Matthias Welter}\\	
\end{tabular}




\chapter*{Thema und Aufgabenstellung}\thispagestyle{empty}
\label{sec:thema}
\pdfbookmark[0]{Thema und Aufgabenstellung}{sec:thema}

\vspace{-0.5cm}
\textbf{Thema:}\\
Horizontale Geschwindigkeitsregelung eines Quadrocopter mit Hilfe von Laserdaten\vspace{0.2 cm}

\noindent\textbf{Aufgabenstellung:}\\
Um das manuelle sowie automatisierte Navigieren eines Quadrocopters in der horizontalen Ebene zu vereinfachen ist es von Vorteil, die Bewegung ausschlie�lich in Form von Geschwindigkeiten in x- und y-Richtung vorzugeben. 
Manuell soll die Vorgabe �ber die Fernsteuerung erfolgen. F�r das automatisierte Navigieren ist eine Schnittstelle zum �bergeben der Sollwerte vorzusehen. Die Geschwindigkeit ist anhand der vom Laserscanner erfassten Daten zu ermitteln.\\
\textbf{\textit{Ziel ist es eine Regelung zu entwerfen, welche die horizontale Geschwindigkeit des Quadrocopters auf den Sollwert einregelt.}}\\	
Optional kann eine automatisierte relative Positionsverschiebung des Quadrocopters implementiert werden.\vspace{0.2 cm}
\\
Die Arbeitsschritte sind:
\begin{itemize}
	\item Literaturrecherche
	\item Auswahl und Integration einer geeigneten Methode zur Bestimmung der relativen Position aus den Laserdaten
	\item Bestimmung der Geschwindigkeit in der x-y-Ebene
	\item Entwurf und Implementierung einer Geschwindigkeitsregelung
	\item Optional: Integration einer automatisierten relativen Positionsverschiebung
	
\end{itemize}
\vspace{0.2 cm}
\textbf{Klassifikation:}\\
Robotik, Regelungstechnik, Informatik, Elektrotechnik, Sensorik
%\includepdf[lastpage=1]{thema_py.pdf} %Aufgabenstellung, geht nur in PDFLatex
\clearpage
\thispagestyle{empty}
\chapter*{Kurzzusammenfassung}
\label{sec:Kurzzusammenfassung}
\pdfbookmark[0]{Kurzzusammenfassung}{sec:Kurzzusammenfassung}

\noindent
Diese Arbeit befasst sich mit dem autonomen anfliegen von Positionen in einem Raum. 

\emph{\textcolor{red}{Hier soll eine kurze Zusammenfassung der Arbeit eingef�gt werden, in der grob umrissen wird, um welches Thema es sich bei der Arbeit dreht und die Ergebnisse, die erzielt worden sind.
Die Kurzzusammenfassung soll nur eine halbe bis dreiviertel Seite lang sein, auf keinen Fall l�nger als eine Seite!}}



\chapter*{Abstract}
\label{sec:Abstract}
\pdfbookmark[0]{Abstract}{sec:Abstract}

\noindent
\emph{\textcolor{red}{Die englische Version der Kurzzusammenfassung. F�r die L�nge gelten die Gleichen Vorgaben wie f�r die deutsche Version.}}




\clearpage
\thispagestyle{empty}
\chapter*{Vorwort}
\thispagestyle{empty}
\label{sec:Vorwort}
\pdfbookmark[0]{Vorwort}{sec:Vorwort}

\textcolor{red}{
\emph{Hier k�nnen allgemeine Hinweise zur Arbeit gegeben werden, bspw. wie man mit englischen Begriffen, Abk�rzungen und Codeabschnitten umgeht. Der nachfolgende Text kann als Beispiel gesehen werden, ist aber keinesfalls verpflichtend und sollte der eigenen Konvention angepasst werden!}}
\medskip
\hrule
\medskip

\noindent 
Diese Arbeit behandelt ein aktuelles technisches Thema, die Verwendung von englischen Begriffen ist unumg�nglich. Soweit sinnvoll findet eine deutsche �bersetzung Verwendung. Nicht �bersetzbare Begriffe, die eine wichtige Bedeutung f�r diese Arbeit haben, werden in einer Fu�note erkl�rt. Ausdr�cke und Bezeichnungen aus Standards werden allgemein nicht �bersetzt. Englische Begriffe sind im Text kursiv geschrieben. 
W�rter, die im deutschen Sprachgebrauch allt�gliche Anwendung finden, wie beispielsweise Computer, Software, Internet etc., sind nicht kursiv geschrieben.

Bei erstmaliger Verwendung von Abk�rzungen wird die volle Bezeichnung ausgeschrieben und das K�rzel dahinter in Klammern gesetzt. In der Folge wird nur die Abk�rzung benutzt.

 
Quelltexte von Programmen sowie programmiertechnische Bezeichnungen und Schl�sselw�rter werden durch die Verwendung von Schreibmaschinenschrift hervorgehoben. 

Am Anfang der Arbeit befindet sich ein Abk�rzungsverzeichnis, in dem alle in dieser Arbeit genannten Abk�rzungen und deren ausgeschriebenen Formen enthalten sind. Im Anschluss an den Ausblick werden die wichtigsten Begriffe im Glossar zus�tzlich kurz erl�utert.

 




\frontmatter

%Glossar und Abk�rzungsverzeichnis sollen wie Kapitel angesehen werden
\setglossarysection{chapter}

%Abk�rzungsverzeichnis ausgeben
\deftranslation[to=German]{Acronyms}{Abk�rzungsverzeichnis}
\printglossary[type=\acronymtype,style=long]

%Symbole ausgeben
\printglossary[type=symbolslist,style=long]

\clearpage %%% ggf. \cleardoublepage
\phantomsection
\pdfbookmark[0]{Inhaltsverzeichnis}{toc}
\tableofcontents

% Hauptteil
\mainmatter
\pagenumbering{gobble}


\clearpage
\pagenumbering{arabic}
\chapter{Einleitung}
\label{chap:Einleitung}


Anmerkungen die in der Einleitung auftauchen sollen.

Positionsregelung wird zur Geschwindigkeitsreglung missbraucht. Soll Position wird einfach aufintegriert.

\chapter{Systembeschreibung des Quadrocopters}
\label{chap:Systemarchitektur}
Zu Beginn wird in diesem Kapitel die grundlegende Funktionsweise des Quadrocopters erl�utert. 
Anschlie�end wird die bei dieser Arbeit zum Einsatz kommende Hardware und die Verkn�pfung der einzelnen Komponenten miteinander dargelegt.  Dabei geht es darum, aufzuzeigen an welchen Stellen Software bereits fest implementiert ist und wo eigene Algorithmen integriert werden k�nnen.



\section{Grundlegende Funktionsweise}
\label{sec:grundFkt}
\label{sec:grundlegenefunktionsweiseQuadrocopter }
Ziel dieses Kapitel ist es ein Verst�ndnis daf�r zu erhalten, wie �ber die gezielte Ansteuerungen der vier Rotoren eine Bewegung des Quadrocopters  hervorgerufen werden kann. Dabei wird auf die wirkenden Kr�fte und Momente eingegangen. Zum leichteren Verst�ndnis wird in diesem Kapitel darauf verzichtet, in die Physik der Schubentwicklung \cite{JanKal13} sowie der Kinematik des Quadrocopter einzusteigen. 

Der Schub der Rotoren ist Anhand der Drehzahl $n_i$ der Rotorbl�tter individuell einstellbar. Damit lassen sich die Kr�fte $ F_i $, die an den Endpunkte der Quadrocopterarme der L�nge $ l $ wirken, vorgeben.
% Somit l�sst sich die Kraft $F_i$ jedes Quadrocopterarmes vorgeben. 
Die Gesamtkraft aller Rotoren ergibt den Schubvektor $S^b$.

\begin{figure}
	\centering
	\includegraphics[width = 0.9\textwidth]{images/funktionsweise_quadrocopter}
	\caption[Momente und Kr�fte an einem Quadrocopter]{Momente und Kr�fte an einem Quadrocopter}
	\label{fig:funktionsprinzip}
\end{figure}

\begin{equation}
S^b = \begin{bmatrix}
S^{b}_x\\
S^{b}_y\\
S^{b}_z\\
\end{bmatrix}
=
\begin{bmatrix}
0\\
0\\
F_1+F_2+F_3+F_4\\
\end{bmatrix}
\label{eq:Schubvektor}
\end{equation}

Damit der Qudrocopter eine Bewegung im Raum vollziehen kann, muss dieser Vektor aus der Vertikalen ausgelenkt werden. Dies wird durch eine �nderung der Lage realisiert. Reduziert man zum Beispiel die Drehzahl $n_1$ und erh�ht gleichzeitig die Drehzahl $n_3$ hat das resultierende Kr�fteungleichgewicht �ber die L�nge $ l $ der Quadrocopterarme ein positives Moment um die $y^b$-Achse zur Folge. Der Quadrocopter dreht sich um die $y^b$-Achse, der Pitch-Winkel �ndert sich. Der Quadrocopter erf�hrt in der horizontalen Ebene des Raums eine Beschleunigung. Das gleiche Prinzip gilt auch f�r den Roll-Winkel, sprich Rotation um die $x^b$-Achse. Hier ist allerdings der Drehzahlenunterschied zwischen $n_2$ und $n_4$ verantwortlich f�r die Rotation.

Eine �nderung der Orientierung um die Hochachse z, sprich �nderung des Yaw-Winkels, l�sst sich ebenfalls �ber Variation der Rotordrehzahlen hervorrufen. Dabei kommt der Effekt zum tragen, dass der Widerstand der umgebende Luft entgegen der Drehrichtung der Motoren eine Kraft auf die Rotorbl�tter einpr�gt. Je nach Drehrichtung der Rotoren wirkt damit ein Moment auf den Quadrocopter.
% Somit �ber die Rotoren Momente auf den Quadrocopter wirken.
Diese Momente, die an den Armen des Quadrocopters angreifen, lassen sich zur Vereinfachung in den Schwerpunkt verschieben. Damit bei gleicher Drehzahl aller Rotorbl�tter ein Momentengleichgewicht herrscht, drehen sich die Motoren eins und drei gegen, die Motoren zwei und vier mit dem Uhrzeigersinn. Um die gew�nschte Rotation zu erzielen, wird die Drehzahl $n_1$ und $n_3$ erh�ht und gleichzeitig  $n_2$ und $n_4$  reduziert. Das Ergebnis ist eine Rotation in positive Richtung.

Zusammenfassen lassen sich die f�r die Rotation um die Quadrocopter-Achsen verantwortlichen Momente $M^{b}_{x,y,z}$ in einem Vektor $M^b$ zusammenfassen.  

\begin{equation}
M^b = \begin{bmatrix}
M^{b}_x\\
M^{b}_y\\
M^{b}_z\\
\end{bmatrix}
=
\begin{bmatrix}
l(F_3-F_1)\\
l(F_2-F_4)\\
M_1-M_2+M_3-M_4\\
\end{bmatrix}
\end{equation}

Aufzukl�ren ist, warum mit einer Erh�hung der Drehzahl auch immer eine Reduzierung des Gegenparts verkn�pft ist. Die Begr�ndung lautet, dass der Schubvektor $S^b$ durch eine Rotation m�glichst wenig beeinflusst werden soll.%, um mit der Gesamtschubvorgabe ganz einfach bestimmt zu werden. 


\section{Aufbau der Hardware}
\label{sec:Hardwareaufbau}
Zum Einsatz kommt der AscTec Pelican der Firma \gls{asctec} \cite{hmpgasctec}. Dieser Quadrocopter ist eine spezielle Entwicklung f�r die Forschung. Seine Turmstruktur erm�glicht eine einfache Integration zus�tzlicher Sensoren und Nutzlasten. Durch die Flexibilit�t im Aufbau ist das Ziel dieses Teilkapitels, einen �berblick zur Position der einzelnen Komponenten zu geben. Begleitend zum Text sind diese in Abbildung \ref{fig:hardwareaufbau} dargestellt.\\

	\begin{figure}
		\centering
		\includegraphics[width = 0.75\textwidth]{images/Hardwareaufbau}
		\caption[Hardwareaufbau des Quadrocopters]{Hardwareaufbau des Quadrocopters}
		\label{fig:hardwareaufbau}
	\end{figure}

F�r jeden der vier mit einem Propeller verbundenen Elektromotoren sind separate Motorcontroller zust�ndig. Diese sorgen daf�r, dass sich die von der \gls{fcu} angeforderten Drehzahlen einstellen.

Die \gls{fcu} ist die zentrale Steuer- und Regeleinheit des Quadrocopters. Sie besitzt zwei ARM7 Prozessoren, einen \gls{llp} und einen \gls{hlp}, zudem verschiedene Kommunikationsschnittstellen (vgl. Kapitel \ref{fig:Kommunikationsstruktur}). Zus�tzlich besitzt \gls{fcu} eine inertiale Messeinheit (engl. \gls{imu}). Diese Einheit wird zur Bewegungsdetektion sowie zur Bestimmung der Lage und Ausrichtung ben�tigt. Sie ist nicht zur Positionsbestimmung in einem ortsfesten Koordinatensystem geeignet. Bestandteile der \gls{imu} sind ein 3D-Beschleunigungssensor, drei Drehratensensoren (Gyros), ein Kompass sowie ein Drucksensor zur Ermittlung der Flugh�he anhand des Luftdrucks. Verbaut sind die Sensoren mit Ausnahme des Kompass direkt auf der Platine ( Abbildung \ref{fig:fcuplatien}).

Da der Einsatzbereich im Indoorbereich liegt, ist der Drucksensor zur H�henbestimmung in geschlossenen R�umen nicht geeignet. Er liefert erst ab einer H�he von $ 5~m $ zuverl�ssige Werte. Daher wurde in einer vorangegangen Arbeit von Jan Kallwies \cite{JanKal13} die Hardware um ein Modul zur Messung der H�he im Indoorbereich erweitert. Auf diesem Modul befinden sich zwei Infrarotsensoren f�r den Nahbereich. Beide zusammen decken einen Bereich von $ 4~cm $ bis $ 142~cm $ ab. Erweitert wird der Messbereich durch einen Ultraschallsensor f�r Entfernungen von bis zu $ 5~m $. Aus diesen drei Sensordaten wird �ber einen Extended-Kalman-Filter die Flugh�he bestimmt. Eine genaue Beschreibung dieses Fusionsfilters kann in der Arbeit von Jan Kallwies \cite{JanKal13} nachgelesen werden. Da in der vorliegenden Arbeit die Navigation in der horizontale Ebene den Schwerpunkt darstellt, wird dieses Modul hier nicht weiter behandelt.

	\begin{figure}
		\centering
		\includegraphics[width = 0.75\textwidth]{images/FCU_Platine}
		\caption[Platine der \gls{fcu}]{Platine der \gls{fcu}}
		\label{fig:fcuplatien}
	\end{figure}
	
Um in der Horizontalen die Navigation zu gew�hrleisten, muss die Position des Flugk�rpers in der xy-Ebene bekannt sein. Da dies, wie schon beschrieben, nicht mit der Inertialsenorik m�glich ist, wurde in die Turmstruktur der Laserscanner UTM-30LX der Firma Hokuyo (Datenblatt im Anhang \ref{anh:datasheet}) integriert. Dieser Scanner hat eine maximale Reichweite von $ 30~m $ und ein Abtastbereich von $ 270� $. Die Umlauffrequenz betr�gt dabei $ 40~Hz $, d.h. alle $ 25~ms $ steht ein neuer Scan zur Verf�gung.    

Damit zur Berechnung der Position sowie der Implementierung weiterer Algorithmen und Funktionen ausreichend Rechenleistung zur Verf�gung steht, befindet sich auf dem Quadrocopter ein zus�tzlicher Odroid-X Mikrocomputer mit einem Quad Core Prozessor mit $ 1.4~GHz $ und $ 1024~MB $ LP-DDR2 Arbeitsspeicher. Au�erdem besitzt diese Entwicklungsplattform sechs USB-Schnittstellen sowie einen $ 10/100~Mbps $ Ethernet-Anschluss.\\


%Nach dem �berblick �ber die im Quadrocopter verbaute Hardware und deren Komponenten, geht das folgende Kapitel \ref{sec:Kommunikationsarchitekur} auf die Implementierung der Intelligenz �ber Software ein.

%Nun sollte man einen �berblick �ber die im Quadrocopter verbauten Komponenten besitzen. Wie die Einheiten untereinander vernetzt sind, darauf wird im folgenden Kapitel \ref{sec:Kommunikationsarchitekur} eingegangen.

     
\section{Softwarearchitektur und Kommunikationsstruktur}
\label{sec:Kommunikationsarchitekur}

Nachdem im vorhergegangenen Kapitel \ref{sec:Hardwareaufbau} die verbaute Hardware vorgestellt wurde, geht es in diesem Abschnitt um die Softwarearchitektur (Abbildung \ref{fig:Kommunikationsstruktur}). Es wird aufgezeigt, welche Software bereits fest implementiert ist und wo adaptive Applikationen integriert werden k�nnen. Des Weiteren wird die Kommunikationsstruktur dargelegt, wie und �ber welche Protokolle die einzelnen Komponenten miteinander kommunizieren. \\
\begin{figure}
	\centering
	\includegraphics[width = \textwidth]{images/Kommunikationsarchitektur}
	\caption[Softwarearchitektur und Kommunikationsstruktur des Quadrocopters]{Softwarearchitektur und Kommunikationsstruktur des Quadrocopters}
	\label{fig:Kommunikationsstruktur}
\end{figure}
Beginnend mit den beiden Prozessoren 
%der \gls{llp} und \gls{hlp} 
der \gls{fcu}, deren Hauptschleifen der Software mit einer Frequenz von 1kHz durchlaufen werden und die 
%mit der \gls{fcu}, deren beiden Prozessoren \gls{llp} und \gls{hlp} mit einer Frequenz von 1kHz getaktet werden und 
�ber einen \gls{spi} Bussystem verkn�pft sind, wird zun�chst der \gls{llp} betrachtet. Auf dem Low Level Prozessor befindet sich die Sensordatenfusion der \gls{imu}-Sensorik zur Lagebestimmung des Quadrocopters. Darauf basiert die Lageregelung, die das Flugverhalten stabilisiert. Hier�ber werden die geforderten Sollwinkel bzw. die Solllage eingestellt, die dem \gls{llp} �ber die Fernbedienung oder den \gls{hlp} �bergeben werden. Kombiniert mit der Schubvorgabe werden den Motorreglern die jeweiligen Solldrehzahlen der Rotoren �ber einen \gls{i2c}-Bus �bergeben. Diese Algorithmen sind fest eingepflegt und gew�hrleisten bei Experimentalfl�gen eine sichere R�ckfallebene. Mit dem \gls{llp} stellt \gls{asctec} dem Benutzer eine Art White-Box zur Verf�gung, d.h. die Integration ist bekannt, jedoch nicht deren Umsetzung. �berwachen l�sst sich der LLP �ber einen externen \gls{pc}, in Abbildung \ref{fig:Kommunikationsstruktur} als Bodenstation bezeichnet. Zur Kommunikation werden zwei XBee Funkmodule ben�tigt. Eines ist am \gls{uart} LL-Serial0 Port der \gls{fcu} angeschlossen, das andere am USB Port der Bodenstation. Mit der AutoPilot Software lassen sich unter anderem der Akkustand, die \gls{imu}-Daten sowie die Stellgr��en der Fernsteuerung betrachten. Au�erdem ist es m�glich, Parameter der Sensorfusion und der Lageregelung auszulesen und zu ver�ndern.


Mit dem \gls{hlp} stellt \gls{asctec} eine Entwicklungsumgebung zur Implementierung eigener Algorithmen auf der \gls{fcu} zur Verf�gung. Hier k�nnen erweiternde Programmteile integriert werden, die den Lageregler des \gls{llp} ansprechen oder die direkt den Motorcontroller �ber den \gls{llp} mit Solldrehzahlen speisen.

Die experimentelle Software auf dem \gls{hlp} kann �ber die Fernbedienung aktiviert und deaktiviert werden. Eine fehlerhafte Programmierung des \gls{hlp} kann kritische Flugman�ver hervorrufen. Damit diese nicht zum Absturz f�hren, kann �ber die Fernbedienung die Experimentalsoftware deaktiviert und das Flugsystem �ber die ausgereifte Lageregelung auf dem \gls{llp} stabilisiert werden (R�ckfallebene).

Wie schon in Kapitel \ref{sec:Hardwareaufbau} beschrieben, befindet sich auf dem Quadorcopter zur Erh�hung der Rechenleistung der Odroid-X. Anders als bei den auf der \gls{fcu} befindlichen Prozessoren, besitzt das Odroid-Bord ein Betriebssystem. Es handelt sich dabei um das opensource Betriebssystem Ubuntu 13.04. Dieses wurde ausgew�hlt, da es die Installation eines weiteren opensource Betriebssystems erm�glicht, dem \gls{ros}, einem Software Framework f�r Roboteranwendungen (Kapitel \ref{sec:ros}). Zum Einsatz kommt der Odroid-X bei der Implementierung der Positionsbestimmung (Kapitel \ref{chap:2Dpositionsbestimmung}). Verbunden ist es zum einen �ber einen USB-Port mit dem Laserscanner. Zum anderen mittels eines weiteren USB-Port �ber einen \gls{ftdi}-Konverter am HL-Serial0 Port des \gls{hlp} angeschlossen.
Von der Bodenstation kann �ber \gls{wlan} eine \gls{ssh}~-~Verbindung aufgebaut werden, die in Folge die Entwicklungsplattform bedient.

Nun ist bekannt, wie die einzelnen Komponenten untereinander vernetzt sind. Im weiteren Verlauf der Arbeit l�sst sich nachvollziehen, an welchen Stellen die Anwendungen implementiert werden und �ber welche Verbindungen sie miteinander kommunizieren. 


\chapter{Grundlagen}
\label{chap:grundlagen}
Das Kapitel Grundlagen behandelt die Themen, die in mehreren Abschnitten dieser Arbeit relevant sind. Dabei handelt es sich um das Robot Operation System, die verwendeten Koordinatensysteme und die Transformation zwischen ihnen.

\section{Das Robot Operation System \gls{ros}}
\label{sec:ros}
Ziel dieses Unterkapitel ist es das Opensource Betriebssystem \gls{ros} vorzustellen. Wie es aufgebaut ist und welche Vorz�ge es besitzt.\\

\gls{ros} stellt dem Softwareentwickler Bibliotheken und Werkzeuge zur Verf�gung, die Helfen Roboteranwendungen zu erstellen. Das auf einem \gls{ip}-basierende  modulare Kommunikationsframework erm�glicht die Verkn�pfung von Anwendungssoftware, Sensoren und Aktoren sogar unter mehreren Robotern. Die Grundlage daf�r ist die sogenannte Hardwareabstraktion. Dabei wird durch hardwarespezifische Module erreicht, das Komponenten unterschiedlicher Hersteller miteinander verbunden werden k�nnen. In unserem Fall Hokuyo Lasersanner und \gls{asctec} \gls{fcu}. Au�erdem erm�glicht es eine hardwareunabh�ngige Programmierung, die  in den Programmiersprachen C/C++ oder in Python erfolgen kann. Jede Hardwareabstraktion oder Anwendung wird als Node, bzw. Konten bezeichnet und l�uft als eigener Prozess.
\begin{figure}
	\centering
	\includegraphics[width = 0.75\textwidth]{images/TopicUndService}
	\caption[Topic und Service]{Kommunikation von Nodes �ber Topics und Services}
	\label{fig:node_kommunikation}
\end{figure}


Der Austausch von Daten zwischen den Nodes erfolgt �ber so genannte Topics (Abbildung \ref{fig:node_kommunikation}]. Dabei werden von den Knoten Nachrichten (engl. Messages) in Topics gepostet und somit ver�ffentlicht (publication). Ben�tigt ein weiter Knoten den Inhalt dieses Topic kann er es abonnieren (subscription). Sobald die Nachricht im Knoten aktualisiert wurde, wird sie den abonnierenden Knoten �bertragen. Dabei sind Knoten nicht auf ein Topic beschr�nkt, es k�nnen beliebig viele Topics beschrieben oder empfangen werden. Alternative zu dieser Art der asynchronen Daten�bertragung, biete \gls{ros} die M�glichkeit einer Synchrone Kommunikation zwischen zwei Nodes �ber Services. Dabei wird auf einem Knoten ein Service gestartet. Dieser dient als Server und agiert nach dem Anfrage-Antwort-Prinzip. Schickt ein anderer Knoten eine Anfrage, wird ihm die geforderte Nachricht zu gesendet. 

Anzumerken ist, das durch das verwendete \gls{ip}-Protokoll keine deterministische Versendung der Nachrichten nicht gew�hrleistet ist, da es sein kann, das Nachrichten gleichen Types in Paketen zusammengefasst werden. Bei der Programmierung empfiehlt es sich daher auf Topics mit einem Zeitstempel (engl. timestamp) zur�ckzugreifen. Die Echtzeitf�higkeit des \gls{ros} ist durch allerdings nicht gef�hrdet. 

\begin{figure}
	\centering
	\includegraphics[width = \textwidth]{images/MasterAndNode}
	\caption[Registrierung der Knoten]{Registrierung der Knoten}
	\label{fig:node_master}
\end{figure}


Der wohl gr��te Vorteil von \gls{ros} ist die st�ndig wachsende Community. So stellen Forscher aus der ganze Welt ihre Algorithmen und Hardwareabstraktionen zur Verf�gung. Dadurch ist es m�glich bei der Erstellung einer Roboteranwendung auf Bausteine zur�ck zugreifen, die ohne diese Plattform selbst zu implementieren w�ren.


\section{Einf�hrung in die Koordinatensysteme und Koordinatentransformationen}
\label{sec:koordinatensysteme&transformationen}

\subsection{Koordinatensysteme}
\label{subsec:koordinatensysteme}
 \begin{figure}
 	\centering
 	\includegraphics[width = \textwidth]{images/Koordinatensysteme}
 	\caption[Koordinatensysteme]{In der Arbeit angewandte Koordinatensysteme }
 	\label{fig:koordinatensysteme}
 \end{figure}
\subsection{Koordinatentransformationen}
\label{subsec:koordinatentransformation}



\chapter[2D Positionsbestimmung]{Zweidimensionale Positionsbestimmung des Quadrocoters in xy-Ebene des Navigationsframes }
\label{chap:2Dpositionsbestimmung}
Wie schon in der Einleitung (Kapitel \ref{chap:Einleitung}) sowie der Aufgabenstellung Beschrieben, erfolgt die Positionsbestimmung �ber den auf dem Quadrocopter montierten Lasersanner. Man spricht hierbei von einem Onboard-Lokalisierungssystem. Die aufgenommen Entfernungen sind dabei im l-frame definiert. Die Rohdaten enthalten somit keine Information �ber die Position des Quadrocopters im Navigationskoordinatensystem, sondern jegentlich die Entfernung von umgebenden Objekten, bzw. W�nden. Anhand derer l�sst sich jedoch �ber die Methode des \glqq scanmatching\grqq  die Position in einer zweidimensionalen Ebene bestimmt werden (Kapitel \ref{sec:laser_scan_matching}). Da diese Ebene der xy-Ebene des n-frames entsprechen soll, m�ssen die Laserdaten zun�chst in das o-frame �berf�hrt werden (Kapitel \ref{sec:laser_ortho}). 


\begin{figure}
	\centering
	\includegraphics[width = .55\textwidth]{images/ros_ortho_scan_node}
	\caption[Verkn�pfung der Scantools]{Verkn�pfung des \glqq laser\_ortho\_projector \grqq - und des \glqq laser\_scan\_matcher\grqq -Knoten GRAFIK EVENTUELL �BERARBEITEN DA ASCTEC\_HL\_INTERFACE KNOTEN DER F�R DIE KOMMUNIKATION VERANTWORTLICH IST FEHLT} %
	\label{fig:verkn_scantools}
\end{figure}

Realisiert sind diese Vorg�nge in den von \gls{ros} zu Verf�gung gestellte scan\_tools.  Genauer gesagt handelt es sich dabei um dem \glqq laser\_ortho\_projector \grqq - und dem  \glqq laser\_scan\_matcher\grqq -Knoten (Abbildung \ref{fig:verkn_scantools}).Ziel der folgenden Kapitel ist es die Mathematik sowie die Funktionsweise die hinter diesen Algorithmen steht zu erl�utern. 	

ANMERKUNG ES FEHLT NOCH EIN LITERATURVERZEICHNIS!	



\section{Projektion der Laserdaten in das o-frame auf der xy-Ebene des n-frames (\glqq laser\_ortho\_projector\grqq) }
\label{sec:laser_ortho}
Bei der Laserprojektion werden die Laserdaten des l-frame orthogonal zur xy-Ebene des n-frames in die des o-frame transformiert. In allgemeiner Form ist dies in Abbildung \ref{fig:laser_proj_allg} dargestellt. M�glich ist diese Art der Projektion nur unter der Annahme, das es sich bei den erfassten Objekten um Gegenst�nde handelt, die Aufgrund ihrer rechtwinkligen Eigenschaften unabh�ngig der H�he in der sie erfasst werden die gleich Formen aufweisen. In geschlossen R�umen ist diese Annahme zutreffend, da es sich bei den Objekten haupts�chlich um W�nde handelt. Durch Erf�llung dieser Voraussetzungen kann die Flugh�he des Quadrocopters vernachl�ssigt werden. Dies kann man aus Abbildung \ref{fig:laser_proj_pi} entnehmen. Eine Verschiebung des Koordinaten Ursprungs des b-frames auf der z-Achse des o-frames hat demzufolge keinen Einfluss auf die Projektion. Folglich kann f�r beide Koordinatensystem der identischen Ursprung angenommen werden. Unter Beachtung dieser Annahmen wird im Folgenden die Transformation der Laserdaten das o-frame dargestellt.\\

  
\begin{figure}
	
	\centering{
		\subfloat[Allgemeine Projektion der Laserdaten]{
			\includegraphics[width=0.5\textwidth]{images/laser_scan_project}
			\label{fig:laser_proj_allg}
		}
		\subfloat[Projektion eines Punkt $P_i$ in den o-frame]{
			\includegraphics[width=0.5\textwidth]{images/laser_ortho_pro_zx_bsp}
			\label{fig:laser_proj_pi}
		}
	}	
	\caption[Laserprojektion]{Projektion der Laserdaten}
	\label{fig:laser_proj}
	
\end{figure}

Die Entfernungsdaten eines Umlaufs besteht aus mehreren diskreten Abtastungen. �bergeben werden sie in Form von Entfernung $r_i$ in einem Array $r$. Mittels der Schrittweite von $0.25^\circ$ l�sst sich anhand des Indizes $i$ jeder Messung einen Winkel $\gamma$ zuweisen. 
\begin{equation}
\gamma_i = 135^\circ - 0.25^\circ \cdot i
\end{equation}
Die Entfernung eines Punktes $P_i$ ist somit �ber $\{r_i, \gamma_i\}$ definiert. Zur weiteren Verwendung ist es notwendig die Messungen im kartesischen Koordinatensystem des l-frame zu �bertragen.

  \begin{equation}
  P_{il} = 
  \begin{bmatrix}
  \cos(\gamma_i)\cdot r_i, & \sin(\gamma_i)\cdot r_i, & 0
  \end{bmatrix}^T
  \end{equation}
  
Da der Bezugspunkt des o-frames im Schwerpunkt des Quadrocoptes liegen soll, in dem auch der b-frame seinen Ursprung hat, ist es von n�ten die Laserdaten vom l-frame ins b-frame zu transformieren. Wie schon in Kapitel \ref{sec:koordinatensysteme&transformationen} erw�hnt handelt es sich dabei um eine Konstante Transformation. Genauer gesagt um einen Offset von $10cm$ auf der $z^b$-Achse, da der Laser Oberhalb des Quadrocopterschwerpunktes montiert ist.

\begin{equation}
P_{ib} = 
\begin{bmatrix}
\cos(\gamma_i)\cdot r_i, & \sin(\gamma_i)\cdot r_i, & 0,1
\end{bmatrix}^T
\end{equation}
 
 %Bemerkungen:
% keine Rotation um yaw
 %Orientierung der x-Achse identisch da zu erst um y-achse gedreht siehe Konvention
 %Transformationsmatrix auf richtung achten
 %�berpr�fung der berechnung der nein gamma winkel des Lasers. Und Sinn dahinter .Wahrschienlich unn�tz

\begin{figure}
	\centering
	\includegraphics[width = .55\textwidth]{images/l_frame}
	\caption[l-frame]{Draufsicht l-frame} %
	\label{fig:l-frame}
\end{figure}

\section{Positionsbestimmung anhand der ins o-frame �berf�hrten Laserdaten �ber scanmatching}
\label{sec:laser_scan_matching}


\printglossary[style=altlist,title=Glossar]

% Anhang (Bibliographie darf im deutschen nicht in den Anhang!)
\bibliography{bib/BibtexDatabase}
\clearpage
%\addcontentsline{toc}{chapter}{Abbildungsverzeichnis}
\listoffigures
\clearpage
%\addcontentsline{toc}{chapter}{Tabellenverzeichnis}
\listoftables

% Anhang
\appendix
% \input{content/Z-Anhang-01-Herleitungen}

\chapter{Datenbl�tter}
\label{chap:Anhang}

FLACHHEIT NICHT LINEAER MEHRGR�?ENSYSTEME VON HERR DEUTSCHER

\chapter{Simulinkmodell}


%% Dokument ENDE %%%%%%%%%%%%%%%%%%%%%%%%%%%%%%%%%%%%%%%%%%%%%%%%%%%%%%%%%%
\end{document}

