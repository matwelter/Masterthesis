\chapter[Zusammenfassung und Ausblick]{Zusammenfassung und Ausblick}
\label{chap:zusammenfassungAusblick}

\section{Zusammenfassung}
\label{sec:zusammenfassung}
Im Rahmen \DIFdelbegin \DIFdel{der Masterthesis , }\DIFdelend \DIFaddbegin \DIFadd{dieser Masterthesis }\DIFaddend wurde eine Positions- und Geschwindigkeitsreglung\DIFdelbegin \DIFdel{basierend auf Onboardlokalisierungssystem }\DIFdelend \DIFaddbegin \DIFadd{, basierend auf einem Onboardlokalisierungssystem, }\DIFaddend mittels eines Laserscanners auf dem Pelican Quadrocopter der Firma \gls{asctec} implementiert. \DIFdelbegin \DIFdel{Dabei sind bei der Lokalisierung als auch f�r Regelung }\DIFdelend \DIFaddbegin \DIFadd{Zur Lokalisierung und Regelung wurden dabei }\DIFaddend bereits entwickelte Funktionsbausteine zu einem funktionierenden Gesamtsystem \DIFdelbegin \DIFdel{zusammen gesetzt worden. So liegt der Schwerpunkt dieser Arbeit darauf }\DIFdelend \DIFaddbegin \DIFadd{zusammengef�gt. So ist der Schwerpunkt darauf ausgelegt, }\DIFaddend die Theorie und \DIFaddbegin \DIFadd{die }\DIFaddend Konzepte dieser Bausteine offenzulegen.

Begonnen wurde mit der Lokalisierung des Quadrocopters. Dort \DIFdelbegin \DIFdel{wir }\DIFdelend \DIFaddbegin \DIFadd{wird }\DIFaddend aufgezeigt, wie der Einfluss der Neigungswinkel auf die Entfernungsmessung \DIFdelbegin \DIFdel{, }\DIFdelend mittels einer orthogonalen Projektion der Messwerte auf die horizontale Ebene des Raumes eliminiert wird. Die Grundvoraussetzung f�r den zur Positionsbestimmung zur Anwendung kommenden Scanmatching-Algorithmus, dessen Funktionsweise \DIFaddbegin \DIFadd{ist }\DIFaddend in dieser Arbeit ebenfalls mathematisch \DIFdelbegin \DIFdel{dargelegt ist}\DIFdelend \DIFaddbegin \DIFadd{abgeleitet}\DIFaddend .

Im n�chsten Schritt erfolgte die Validierung der von \gls{asctec} in Verbindung mit der ETH-Z�rich entwickelten Positionsregelung. Angelehnt an die Funktionsbeschreibung des Reglers in \cite{Achtelik11} \DIFdelbegin \DIFdel{, }\DIFdelend bestand die Herausforderung darin, den dort beschrieben Aufbau der Regelung mittels Literaturverweise, der Herleitung der verwendeten Regelalgorithmen sowie anhand von Simulationsbeispielen zu legitimieren. Zus�tzlich \DIFdelbegin \DIFdel{wird aufgezeigt}\DIFdelend \DIFaddbegin \DIFadd{wurde aufgezeigt, }\DIFaddend �ber welche Stellschrauben die Komponenten der Regelung zu parametrieren sind.

Nachdem Legitimierung der \DIFdelbegin \DIFdel{Positionregelung }\DIFdelend \DIFaddbegin \DIFadd{Positionsregelung }\DIFaddend mit einer abschlie�enden Simulation des Gesamtsystems in der Theorie bewiesen war, erfolgten Flugversuche mit dem Quadrocopter. Diese belegten die \DIFdelbegin \DIFdel{Funktionsf�hig }\DIFdelend \DIFaddbegin \DIFadd{Funktionsf�higkeit }\DIFaddend der Positionsregelung als auch der Geschwindigkeitsregelung in der Praxis, was auch aus den in dieser Arbeit ver�ffentlichen Messdaten zweier Flugversuchen hervorgeht. Allerdings stellte sich dabei heraus, \DIFdelbegin \DIFdel{das }\DIFdelend \DIFaddbegin \DIFadd{dass }\DIFaddend es kleine Defizite bei der Positionsbestimmung mittels des Lasers zu beheben gibt. \DIFdelbegin \DIFdel{So kommt es vor, das immer mal wieder }\DIFdelend \DIFaddbegin \DIFadd{Vereinzelt zeigte sich, dass }\DIFaddend �ber kurze Zeitr�ume von circa eine Sekunde die Position des Quadrocopters nicht aktualisiert \DIFdelbegin \DIFdel{wird. Dies f�hrt }\DIFdelend \DIFaddbegin \DIFadd{werden konnte. Dies f�hrte }\DIFaddend zu einer Fehlinterpretation der Positionsregelung �ber den Zustand des Quadrocopters, was \DIFdelbegin \DIFdel{in ung�nstigen }\DIFdelend \DIFaddbegin \DIFadd{im ung�nstigsten }\DIFaddend Fall zum Absturz des Flugobjektes f�hren kann. Hier gilt es im Anschluss dieser Arbeit \DIFdelbegin \DIFdel{Nachforschungen }\DIFdelend \DIFaddbegin \DIFadd{um Nachforschungen, die }\DIFaddend zur Ergr�ndung der Ursache dieser Positionsfehlstellen \DIFdelbegin \DIFdel{durchzuf�hren}\DIFdelend \DIFaddbegin \DIFadd{f�hren k�nnen}\DIFaddend . 

\section{Ausblick}
\label{sec:ausblick}
\DIFdelbegin \DIFdel{Trotz }\DIFdelend \DIFaddbegin \DIFadd{Ungeachtet }\DIFaddend der aktuell bekannten Defizite bei der Positionsbestimmung \DIFdelbegin \DIFdel{, }\DIFdelend ist die Empfehlung zun�chst die H�hensch�tzung von Jan Kallwies \cite{JanKal13} auf dem Quadrocopter zu integrieren. Dies w�rde nicht nur dazu f�hren, dass mittels der Positionsregelung  Koordinaten in einem dreidimensionalen Raum vollst�ndig autonom angeflogen werden k�nnen, sondern er�ffnet gleichzeitig neue M�glichkeiten bei Positionsbestimmung mittels des Lasers. So kann unter Ber�cksichtigung der Flugh�he zur Bestimmung der Position ein 3D-\gls{slam}-Algortihmus eingesetzt werden. Diese Art der Lokalisierung hat gegen�ber des simplen Scanmatching-Verfahren den Vorteil, \DIFdelbegin \DIFdel{das }\DIFdelend \DIFaddbegin \DIFadd{dass }\DIFaddend es unabh�ngig von senkrechten Eigenschaften der Umgebung ist. Ob dies jedoch dazuf�hrt das keine L�cken in der Positionsbestimmung auftreten\DIFaddbegin \DIFadd{, }\DIFaddend ist nicht garantiert. Deshalb gilt es, sich zus�tzlich mit der Entwicklung eines Notlaufprogramms zu besch�ftigen. Dieses k�nnte so aufgebaut sein, \DIFdelbegin \DIFdel{das }\DIFdelend \DIFaddbegin \DIFadd{dass }\DIFaddend es ausbleibende Aktualisierungen der Positionsdaten erkennt, alle von der Regelung geforderten Stellwinkel zur�ck nimmt und erst nach \DIFdelbegin \DIFdel{erhalten neuer Positionswerte }\DIFdelend \DIFaddbegin \DIFadd{Erhalten neuer Positionswerte diese }\DIFaddend wieder freigibt. Gleichzeitig w�re eine akustische Warnung denkbar, so dass der Bediener im \DIFdelbegin \DIFdel{Erstfall }\DIFdelend \DIFaddbegin \DIFadd{Ernstfall }\DIFaddend die Regelung komplett deaktivieren und das Flugsystem rechtzeitig manuell stabilisieren kann.

Alles in allem ist \DIFdelbegin \DIFdel{, }\DIFdelend das Ziel mit dem Quadrocopter einer autonome \DIFdelbegin \DIFdel{Refrenzplattform }\DIFdelend \DIFaddbegin \DIFadd{Referenzplattform }\DIFaddend f�r \DIFaddbegin \DIFadd{das }\DIFaddend am Fraunhofer-Insitut f�r Integrierte Schaltungen entwickelte Lokalisierungssysteme  durch diese Arbeit \DIFaddbegin \DIFadd{in den Erkenntnissen }\DIFaddend ein St�ck n�her ger�ckt.
