%%%
%Symbole
%%%
\newglossaryentry{symb:Pi}{
name=$\pi$,
description={Die Kreiszahl.},
sort=symbolpi, type=symbolslist
}
\newglossaryentry{symb:Phi}{
name=$\varphi$,
description={Ein beliebiger Winkel.},
sort=symbolphi, type=symbolslist
}
\newglossaryentry{symb:Lambda}{
name=$\lambda$,
description={Eine beliebige Zahl, mit der der nachfolgende Ausdruck
multipliziert wird.},
sort=symbollambda, type=symbolslist
}

%%%
%Abk�rzungen
%%%
\newacronym{FCU}{FCU}{Flight-Contorl-Unit}
\newacronym{MS}{MS}{Microsoft}
\newacronym{CD}{CD}{Compact Disc}

%Eine Abk�rzung mit Glossareintrag
\newacronym{AD}{AD}{Active Directory\protect\glsadd{glos:AD}}


%%%
%Glossareintr�ge
%%%
\newglossaryentry{glos:AD}{
name=Active Directory,
description={Active Directory ist in einem Windows 2000/" "Windows
Server 2003-Netzwerk der Verzeichnisdienst, der die zentrale
Organisation und Verwaltung aller Netzwerkressourcen erlaubt. Es
erm�glicht den Benutzern �ber eine einzige zentrale Anmeldung den
Zugriff auf alle Ressourcen und den Administratoren die zentral
organisierte Verwaltung, transparent von der Netzwerktopologie und
den eingesetzten Netzwerkprotokollen. Das daf�r ben�tigte
Betriebssystem ist entweder Windows 2000 Server oder
Windows Server 2003, welches auf dem zentralen
Dom�nencontroller installiert wird. Dieser h�lt alle Daten des
Active Directory vor, wie z.B. Benutzernamen und
Kennw�rter.}
}
\newglossaryentry{glos:AntwD}{name=Antwortdatei, description={Informationen zum
Installieren einer Anwendung oder des Betriebssystems.}}
