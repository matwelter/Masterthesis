%%%
%Symbole
%%%
\newglossaryentry{symb:Pi}{
name=$\pi$,
description={Die Kreiszahl.},
sort=symbolpi, type=symbolslist
}
\newglossaryentry{symb:Phi}{
name=$\varphi$,
description={Ein beliebiger Winkel.},
sort=symbolphi, type=symbolslist
}
\newglossaryentry{symb:Lambda}{
name=$\lambda$,
description={Eine beliebige Zahl, mit der der nachfolgende Ausdruck
multipliziert wird.},
sort=symbollambda, type=symbolslist
}

%%%
%Abk�rzungen
%%%
\newacronym{fcu}{FCU}{Flight Contorl Unit}
\newacronym{imu}{IMU}{Inertial Measurement Unit}
\newacronym{llp}{LLP}{Low Level Processor}
\newacronym{hlp}{HLP}{High Level Processor}
\newacronym{uav}{UAV}{Unmanned Aerial Vehicle}
\newacronym{ros}{ROS}{Robot Operation System}
\newacronym{asctec}{AscTec}{ASCENDING TECHNOLOGIES}


\newacronym{pc}{PC}{Personal Computer}


%Eine Abk�rzung mit Glossareintrag
\newacronym{spi}{SPI}{Serial Peripheral Interface \protect\glsadd{glos:spi}}
\newacronym{uart}{UART}{Universal Asynchronous Receiver/Transmitter\protect\glsadd{glos:uart}}
\newacronym{i2c}{I $^2$C}{Inter Integrated Circuit\protect\glsadd{glos:i2c}}
\newacronym{ip}{IP}{Internet Protocol\protect\glsadd{glos:ip}}

%%%
%Glossareintr�ge
%%%
\newglossaryentry{glos:spi}{
	name=Serial Peripherial Interface,
	description={Ein von Motorola entwickelter Standard eines synchronen seriellen Datenbus zur Vernetzung zweier digitaler Schaltungen nach dem Master-Slave-Prinzip}}
\newglossaryentry{glos:uart}{
	name= Asynchronous Receiver/Transmitter,
	description={Eine g�ngige serielle Schnittstelle zum Senden und Empfangen von Daten �ber eine Datenleitung. Bildet den Standard der seriellen Schnittstellen an \gls{PC}s und Mikrocontrollern.}}
\newglossaryentry{glos:i2c}{
	name= Inter Integrated Circuit,
	description={Von Philips Semiconductor entwickelter serielles Bussystem. Haupts�chlich zur ger�tinternen Kommunikation zwischen verschieden Schaltungsteilen eingesetzt.}}
\newglossaryentry{glos:ip}{
	name= Internet Protocol,
	description={Ist das am weitverbreitete Netzwerkprotokoll in Computernetzen und stellt die Grundlage des Internets dar}}
