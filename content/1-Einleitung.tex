\chapter{Einleitung}
\label{chap:Einleitung}
Da die Rotoren von Quadrocoptern sowohl zum erzeugen des Auftrieb als auch f�r den Vortrieb genutzt werden, geh�ren sie zur Kategorie der Hubschrauber. Im Gegensatz zu einrotorigen Maschinen ben�tigen sie keine mechanischen Komponenten wie eine Taumelscheibe um die Schubkraft, beziehungsweise die Neigung des Flugger�tes zu variieren. Beim Quadrocopter wird dies durch Drehzahl�nderungen der sich an den Au�enseiten befindlichen Rotoren realisiert. Der mechanische Aufbau ist somit deutlich primitiver. Allerdings steigt auf der elektrischen Seite der regelungstechnische Aufwand um das Flugsystem stabil fliegen zu k�nnen. Durch den technischen Fortschritt stehen jedoch kosteng�nstige Sensoren und Prozessoren mit aussreichender Rechenleistung zur Verf�gung um die Lage des Quadrocopters zu stabilisieren.   

Der Grund warum sich der Quadrocopter in der Kategorie unbemannterflug Ger�te wachsender Beliebtheit erfreut. So werden sie heutzutage unter anderem beim Film eingesetzt, wo sie spektakul�re Luftaufnahmen liefern. Au�erdem finden sie in der Katastrophenhilfe Anwendung. So kann zum Beispiel ohne Gef�hrdung der Retter ein einsturzgef�hrdetes Geb�ude durchsucht werden.

Ein Ziel der Forschung ist es, das der Quadrocopter solche Fl�ge zunehmend autonom durchf�hren kann. Und hier setzt die Arbeit an. Mittels eines  Laserscanner soll die Position des Flugger�ts innerhalb eines geschlossenen Raumes erfasst werden. Darauf aufbauend sollen Koordinaten vorgegeben werden die Quadrocopter selbst�ndig anfliegt. Auch kann eine solche Positionsreglung zur Geschwindigkeitsregelung missbraucht werden, was die Steuerung �ber die Fernbedienung vereinfacht.

Eingesetz soll der Quadrocopter sp�ter am Fraunhofer-Insitut f�r Integrierte Schaltungen als bewegliches Referenzsystem f�r die dort entwickelte Lokalisierungsl�sungen mittels \gls{rfid} oder \gls{wlan}. Durch die Automatisierung lassen sich dynamische Bewegungsabl�ufe zuverl�ssig reproduzieren. So k�nnen jederzeit vergleichbare Messungen durchgef�hrt werden.

     
\section{Ausgangssituation}
\label{sec:Ausgangssituation}
Bei den Literaturrecherchen zu beginn der Arbeit stellte sich heraus, das f�r den verwendeten Quadrocopter Pelican der Firma \gls{asctec} ein frei verf�gbares Framework existiert, welches die wesentlichen Punkte der Aufgabenstellung (siehe \glqq Thema und Aufgabenstellung\grqq) mit sich bringt. So beinhaltet es eine Positionsregelung, welche auch als Geschwindigkeitsregelung genutzt werden kann. Zus�tzlich verf�gt es �ber einen Fusionsfilter zur Interpolation eingespeister niederfrequenter Positionsdaten, in dem diese mit den Beschleunigungswerten vereinigt werden.   


\section{Aufbau und Schwerpunkte der Arbeit}
\label{sec:Aufb&Schwpkt}

Anmerkungen die in der Einleitung auftauchen sollen.

Positionsregelung wird zur Geschwindigkeitsreglung missbraucht. Soll Position wird einfach aufintegriert.
