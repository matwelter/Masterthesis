\chapter*{Thema und Aufgabenstellung}\thispagestyle{empty}
\label{sec:thema}
\pdfbookmark[0]{Thema und Aufgabenstellung}{sec:thema}

\vspace{-0.5cm}
\textbf{Thema:}\\
Horizontale Geschwindigkeitsregelung eines Quadrocopter mit Hilfe von Laserdaten\vspace{0.2 cm}

\noindent\textbf{Aufgabenstellung:}\\
Um das manuelle sowie automatisierte Navigieren eines Quadrocopters in der horizontalen Ebene zu vereinfachen, ist es von Vorteil die Bewegung ausschlie�lich in Form von Geschwindigkeiten in x- und y-Richtung vorzugeben. 
Manuell soll die Vorgabe �ber die Fernsteuerung erfolgen. F�r das automatisierte Navigieren ist eine Schnittstelle zum �bergeben der Sollwerte vorzusehen. Die Geschwindigkeit ist anhand der vom Laserscanner erfassten Daten zu ermitteln.\\
\textbf{\textit{Ziel ist es eine Regelung zu entwerfen, welche die horizontale Geschwindigkeit des Quadrocopters auf den Sollwert einregelt.}}\\	
Optional kann eine automatisierte relative Positionsverschiebung des Quadrocopters implementiert werden.\vspace{0.2 cm}
\\
Die Arbeitsschritte sind:
\begin{itemize}
	\item Literaturrecherche
	\item Auswahl und Integration einer geeigneten Methode zur Bestimmung der relativen Position aus den Laserdaten
	\item Bestimmung der Geschwindigkeit in der x-y-Ebene
	\item Entwurf und Implementierung einer Geschwindigkeitsregelung
	\item Optional: Integration einer automatisierten relativen Positionsverschiebung
	
\end{itemize}
\vspace{0.2 cm}
\textbf{Klassifikation:}\\
Robotik, Regelungstechnik, Informatik, Elektrotechnik, Sensorik