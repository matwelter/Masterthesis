\chapter[Flugversuche]{Verifizierung der Positionsregelung am realen System}
\label{chap:Flugversuche}
Nach dem die Funktionsf�higkeit der von \gls{asctec} und ETH-Z�rich entworfenen Positionsregelung mittels Herleitung der einzelnen Komponenten sowie einer abschlie�enden Simulation des Gesamtsystems (Kapitel \ref{sec:sim_posreg}) in der Theorie beweisen ist, fokussiert sich diese Kapitel \ref{chap:Flugversuche} auf die Verifizierung der Regelung am realen System. So gilt es zu darauf zu achten ob die Zeitkonstante der Lagerregelung, die in der Modellbildung vernachl�ssigt worden ist, eine Auswirkung auf die Positionsregelung hat. Auch zeigen wird sich ob die Positionsbestimmung mittels eines Lasers anhand der in Kapitel \ref{chap:2Dpositionsbestimmung} beschrieben orthogonalen Projektion sowie das scanmatching-Verfahren anwendbar ist. 

Daf�r wird in Kapitel \ref{sec:anflugKoor} eine Messung untersucht, bei der der Positionsregelung eine neue Koordinate �bergeben wird, in die der Quadrocopter �berf�hrt werden soll. Im Anschlie�end Kapitel \ref{sec:geschwReg} wird Bezugnehmend auf das urspr�ngliche Thema der Arbeit, der Geschwindigkeitsregelung des Quadrocopter, die Funktionsf�higkeit der Positionsregelung als eben diese Geschwindigkeitsregelung dargelegt.



\section{Anflug einer Koordinate im Raum}
\label{sec:anflugKoor}


\section{Geschwindigkeitsregelung }
\label{sec:geschwReg}





 