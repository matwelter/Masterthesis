\clearpage
\thispagestyle{empty}
\chapter*{Kurzzusammenfassung}
\label{sec:Kurzzusammenfassung}
\pdfbookmark[0]{Kurzzusammenfassung}{sec:Kurzzusammenfassung}

\noindent
\emph{\textcolor{red}{Hier soll eine kurze Zusammenfassung der Arbeit eingef�gt werden, in der grob umrissen wird, um welches Thema es sich bei der Arbeit dreht und die Ergebnisse, die erzielt worden sind.
Die Kurzzusammenfassung soll nur eine halbe bis dreiviertel Seite lang sein, auf keinen Fall l�nger als eine Seite!}}



\chapter*{Abstract}
\label{sec:Abstract}
\pdfbookmark[0]{Abstract}{sec:Abstract}

\noindent
\emph{\textcolor{red}{Die englische Version der Kurzzusammenfassung. F�r die L�nge gelten die Gleichen Vorgaben wie f�r die deutsche Version.}}




\clearpage
\thispagestyle{empty}
\chapter*{Vorwort}
\thispagestyle{empty}
\label{sec:Vorwort}
\pdfbookmark[0]{Vorwort}{sec:Vorwort}

\textcolor{red}{
\emph{Hier k�nnen allgemeine Hinweise zur Arbeit gegeben werden, bspw. wie man mit englischen Begriffen, Abk�rzungen und Codeabschnitten umgeht. Der nachfolgende Text kann als Beispiel gesehen werden, ist aber keinesfalls verpflichtend und sollte der eigenen Konvention angepasst werden!}}
\medskip
\hrule
\medskip

\noindent 
Da sich diese Arbeit um ein aktuelles technisches Thema dreht, ist die Verwendung von englischen Begriffen unumg�nglich. Es wurde soweit wie m�glich versucht, f�r englische Begriffe eine sinnvolle deutsche �bersetzung zu finden und diese stattdessen zu verwenden. Bei Ausdr�cken, bei denen dies nicht m�glich war, die aber eine wichtige Bedeutung f�r diese Arbeit haben, wird mit einer Fu�note eine kurze Erkl�rung gegeben. Begriffe und Bezeichnungen aus den Standards wurden allgemein nicht �bersetzt. Englische Begriffe sind im Text kursiv geschrieben. W�rter, die inzwischen in den allt�glichen Gebrauch der deutschen Sprache eingeflossen sind, wie beispielsweise Computer, Software, Internet etc., werden nicht kursiv geschrieben. 

Bei Abk�rzungen wird bei der ersten Nennung die volle Bezeichnung ausgeschrieben und die Abk�rzung dahinter in Klammern gesetzt. Im Folgenden wird dann nur noch die Abk�rzung verwendet. 

Quelltexte von Programmen sowie programmiertechnische Bezeichnungen und Schl�sselw�rter werden durch die Verwendung von Schreibmaschinenschrift hervorgehoben. 

Am Anfang der Arbeit findet sich ein Abk�rzungsverzeichnis, in dem alle in dieser Arbeit genannten Abk�rzungen und deren ausgeschriebene Formen enthalten sind. Zus�tzlich befindet sich im Anschluss an den Ausblick ein Glossar, das die wichtigsten Begriffe nochmals kurz erl�utert.