\clearpage
\thispagestyle{empty}
\chapter*{Kurzzusammenfassung}
\label{sec:Kurzzusammenfassung}
\pdfbookmark[0]{Kurzzusammenfassung}{sec:Kurzzusammenfassung}

\noindent
Ziel aktueller Forschungen ist die teilweise bzw. vollst�ndig autonome Fortbewegung von Quadrocoptern. Diese Arbeit befasst sich mit der Positions- und Geschwindigkeitsregelung eines solchen Flugger�tes, basierend auf Entfernungsmessungen, die mittels eines auf dem Quadrocopter befindlichen Laserscanners aufgenommen werden. Zur Lokalisierung und Regelung werden bereits entwickelte Funktionsbausteine zu einem funktionierenden Gesamtsystem zusammengef�gt. So ist der Schwerpunkt darauf ausgelegt, die Theorie und die Konzepte dieser Bausteine offenzulegen und nachzuweisen.

 Dabei wird mit der Lokalisierung. Dort ist gezeigt wie sich trotz eines st�ndig wechselnden Neigungswinkel des Quadrocopter, anhand der Messungdaten des Lasers die Position in der Ebene einer unbekannten Umgebung bestimmen l�sst. Anschlie�end fokussierte sich die Arbeit auf die Verifizierung der implementierten Regelung. Zun�chst wird das kinematische Modell hergeleitet. Darauf aufbauend ist die Korrektheit der von den Entwicklern der Positions- und Geschwindigkeitsregelung ver�ffentlichen Formel und Annahmen f�r die Bestandteile der Regelung durch die mathematische Herleitung, Teilsimulationen sowie Literaturverweise dargelegt.

Zu guter Letzt wird die Funktionsf�higkeit der Regelung in Verbindung  mit der �ber den Laser bestimmten Positionsdaten mittels Flugversuchen, sowohl f�r die Positionsregelung als auch f�r die Geschwindigkeitsregelung, unter Beweis gestellt. 





%\emph{\textcolor{red}{Hier soll eine kurze Zusammenfassung der Arbeit eingef�gt werden, in der grob umrissen wird, um welches Thema es sich bei der Arbeit dreht und die Ergebnisse, die erzielt worden sind.
%Die Kurzzusammenfassung soll nur eine halbe bis dreiviertel Seite lang sein, auf keinen Fall l�nger als eine Seite!}}



\chapter*{Abstract}
\label{sec:Abstract}
\pdfbookmark[0]{Abstract}{sec:Abstract}

\noindent
\emph{\textcolor{red}{Die englische Version der Kurzzusammenfassung. F�r die L�nge gelten die Gleichen Vorgaben wie f�r die deutsche Version.}}




\clearpage
\thispagestyle{empty}
\chapter*{Vorwort}
\thispagestyle{empty}
\label{sec:Vorwort}
\pdfbookmark[0]{Vorwort}{sec:Vorwort}

\textcolor{red}{
\emph{Hier k�nnen allgemeine Hinweise zur Arbeit gegeben werden, bspw. wie man mit englischen Begriffen, Abk�rzungen und Codeabschnitten umgeht. Der nachfolgende Text kann als Beispiel gesehen werden, ist aber keinesfalls verpflichtend und sollte der eigenen Konvention angepasst werden!}}
\medskip
\hrule
\medskip

\noindent 
Diese Arbeit behandelt ein aktuelles technisches Thema, die Verwendung von englischen Begriffen ist unumg�nglich. Soweit sinnvoll findet eine deutsche �bersetzung Verwendung. Nicht �bersetzbare Begriffe, die eine wichtige Bedeutung f�r diese Arbeit haben, werden in einer Fu�note erkl�rt. Ausdr�cke und Bezeichnungen aus Standards werden allgemein nicht �bersetzt. Englische Begriffe sind im Text kursiv geschrieben. 
W�rter, die im deutschen Sprachgebrauch allt�gliche Anwendung finden, wie beispielsweise Computer, Software, Internet etc., sind nicht kursiv geschrieben.

Bei erstmaliger Verwendung von Abk�rzungen wird die volle Bezeichnung ausgeschrieben und das K�rzel dahinter in Klammern gesetzt. In der Folge wird nur die Abk�rzung benutzt.

 
Quelltexte von Programmen sowie programmiertechnische Bezeichnungen und Schl�sselw�rter werden durch die Verwendung von Schreibmaschinenschrift hervorgehoben. 

Am Anfang der Arbeit befindet sich ein Abk�rzungsverzeichnis, in dem alle in dieser Arbeit genannten Abk�rzungen und deren ausgeschriebenen Formen enthalten sind. Im Anschluss an den Ausblick werden die wichtigsten Begriffe im Glossar zus�tzlich kurz erl�utert.

 



