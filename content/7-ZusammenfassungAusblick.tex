\chapter[Zusammenfassung und Ausblick]{Zusammenfassung und Ausblick}
\label{chap:zusammenfassungAusblick}

\section{Zusammenfassung}
\label{sec:zusammenfassung}
Im Rahmen dieser Masterthesis wurde eine Positions- und Geschwindigkeitsreglung, basierend auf einem Onboardlokalisierungssystem, mittels eines Laserscanners auf dem Pelican Quadrocopter der Firma \gls{asctec} implementiert. Zur Lokalisierung und Regelung wurden dabei bereits entwickelte Funktionsbausteine zu einem funktionierenden Gesamtsystem zusammengef�gt. So ist der Schwerpunkt darauf ausgelegt, die Theorie und die Konzepte dieser Bausteine offenzulegen.

Begonnen wurde mit der Lokalisierung des Quadrocopters. Hier ist aufgezeigt worden, wie der Einfluss der Neigungswinkel auf die Entfernungsmessung mittels einer orthogonalen Projektion der Messwerte auf die horizontale Ebene des Raumes eliminiert werden. Die Grundvoraussetzung f�r den bei Positionsbestimmung zum Einsatz kommenden Scanmatching-Algorithmus, dessen Funktionsweise in dieser Arbeit ebenfalls mathematisch dargelegt ist.

Im n�chsten Schritt erfolgte die Validierung der von \gls{asctec} in Verbindung mit der ETH-Z�rich entwickelten Positionsregelung. Angelehnt an die Funktionsbeschreibung des Reglers in \cite{Achtelik11} bestand die Herausforderung darin, den dort beschriebenen Aufbau der Regelung mittels Literaturverweise, der Herleitung der verwendeten Regelalgorithmen sowie anhand von Simulationsbeispielen zu legitimieren. Zus�tzlich wurde aufgezeigt, �ber welche Stellschrauben die Komponenten der Regelung zu parametrieren sind.

Nachdem die Legitimierung der Positionsregelung mittels einer abschlie�enden Simulation des Gesamtsystems in der Theorie bewiesen war, erfolgten Flugversuche mit dem Quadrocopter. Diese belegten die Funktionsf�higkeit der Positionsregelung als auch der Geschwindigkeitsregelung in der Praxis, was auch aus den in dieser Arbeit ver�ffentlichen Messdaten zweier Flugversuche hervorgeht. Allerdings stellte sich heraus, dass es kleine Defizite bei der Positionsbestimmung mittels des Lasers zu beheben gilt. Vereinzelt zeigte sich, dass �ber kurze Zeitr�ume von circa einer Sekunde die Position des Quadrocopters nicht aktualisiert wird. Dies f�hrte zu einer Fehlinterpretation der Positionsregelung �ber den Zustand des Quadrocopters, was im ung�nstigsten Fall zum Absturz des Flugobjektes f�hren kann. Hier gilt es im Anschluss dieser Arbeit Nachforschungen zur Ergr�ndung der Ursache dieser Positionsfehlstellen anzustellen. 

\section{Ausblick}
\label{sec:ausblick}
Ungeachtet der aktuell bekannten Defizite bei der Positionsbestimmung ist die Empfehlung zun�chst die H�hensch�tzung von Jan Kallwies \cite{JanKal13} auf dem Quadrocopter zu integrieren. Dies w�rde nicht nur dazu f�hren, dass mittels der Positionsregelung  Koordinaten in einem dreidimensionalen Raum vollst�ndig autonom angeflogen werden k�nnen, sondern er�ffnet gleichzeitig neue M�glichkeiten bei Positionsbestimmung mittels des Lasers. So kann unter Ber�cksichtigung der Flugh�he zur Bestimmung der Position ein 3D-\gls{slam}-Algortihmus eingesetzt werden. Diese Art der Lokalisierung hat gegen�ber des Scanmatching-Verfahrens den Vorteil, dass es unabh�ngig von senkrechten Eigenschaften der Umgebung ist. Ob dies jedoch dazu f�hrt, dass keine L�cken in der Positionsbestimmung auftreten, ist nicht garantiert. Deshalb gilt es, sich zus�tzlich mit der Entwicklung eines Notlaufprogramms zu besch�ftigen. Dieses k�nnte so aufgebaut sein, dass es ausbleibende Aktualisierungen der Positionsdaten erkennt, alle von der Regelung geforderten Stellwinkel zur�cknimmt und erst nach Erhalten neuer Positionswerte diese wieder freigibt. Gleichzeitig w�re eine akustische Warnung denkbar, so dass der Bediener im Ernstfall die Regelung komplett deaktivieren und das Flugsystem rechtzeitig manuell stabilisieren kann.

Alles in allem ist das Ziel mit dem Quadrocopter eine autonome Referenzplattform f�r das am Fraunhofer-Insitut f�r Integrierte Schaltungen entwickelte Lokalisierungssysteme durch diese Masterthesis ein wesentliches St�ck n�her ger�ckt.
