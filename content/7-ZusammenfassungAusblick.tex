\chapter[Zusammenfassung und Ausblick]{Zusammenfassung und Ausblick}
\label{chap:zusammenfassungAusblick}

\section{Zusammenfassung}
\label{sec:zusammenfassung}
Im Rahmen der Masterthesis, wurde eine Positions- und Geschwindigkeitsreglung basierend auf Onboardlokalisierungssystem mittels eines Laserscanners auf dem Pelican Quadrocopter der Firma \gls{asctec} implementiert. Dabei sind bei der Lokalisierung als auch f�r Regelung bereits entwickelte Funktionsbausteine zu einem funktionierenden Gesamtsystem zusammen gesetzt worden. So liegt der Schwerpunkt dieser Arbeit darauf die Theorie und Konzepte dieser Bausteine offenzulegen.

Begonnen wurde mit der Lokalisierung des Quadrocopters. Dort wir aufgezeigt, wie der Einfluss der Neigungswinkel auf die Entfernungsmessung, mittels einer orthogonalen Projektion der Messwerte auf die horizontale Ebene des Raumes eliminiert wird. Die Grundvoraussetzung f�r den zur Positionsbestimmung zur Anwendung kommenden Scanmatching-Algorithmus, dessen Funktionsweise in dieser Arbeit ebenfalls mathematisch dargelegt ist.

Im n�chsten Schritt erfolgte die Validierung der von \gls{asctec} in Verbindung mit der ETH-Z�rich entwickelten Positionsregelung. Angelehnt an die Funktionsbeschreibung des Reglers in \cite{Achtelik11}, bestand die Herausforderung darin, den dort beschrieben Aufbau der Regelung mittels Literaturverweise, der Herleitung der verwendeten Regelalgorithmen sowie anhand von Simulationsbeispielen zu legitimieren. Zus�tzlich wird aufgezeigt �ber welche Stellschrauben die Komponenten der Regelung zu parametrieren sind.

Nachdem Legitimierung der Positionregelung mit einer abschlie�enden Simulation des Gesamtsystems in der Theorie bewiesen war, erfolgten Flugversuche mit dem Quadrocopter. Diese belegten die Funktionsf�hig der Positionsregelung als auch der Geschwindigkeitsregelung in der Praxis, was auch aus den in dieser Arbeit ver�ffentlichen Messdaten zweier Flugversuchen hervorgeht. Allerdings stellte sich dabei heraus, das es kleine Defizite bei der Positionsbestimmung mittels des Lasers zu beheben gibt. So kommt es vor, das immer mal wieder �ber kurze Zeitr�ume von circa eine Sekunde die Position des Quadrocopters nicht aktualisiert wird. Dies f�hrt zu einer Fehlinterpretation der Positionsregelung �ber den Zustand des Quadrocopters, was in ung�nstigen Fall zum Absturz des Flugobjektes f�hren kann. Hier gilt es im Anschluss dieser Arbeit Nachforschungen zur Ergr�ndung der Ursache dieser Positionsfehlstellen durchzuf�hren. 

\section{Ausblick}
\label{sec:ausblick}
Trotz der aktuell bekannten Defizite bei der Positionsbestimmung, ist die Empfehlung zun�chst die H�hensch�tzung von Jan Kallwies \cite{JanKal13} auf dem Quadrocopter zu integrieren. Dies w�rde nicht nur dazu f�hren, dass mittels der Positionsregelung  Koordinaten in einem dreidimensionalen Raum vollst�ndig autonom angeflogen werden k�nnen, sondern er�ffnet gleichzeitig neue M�glichkeiten bei Positionsbestimmung mittels des Lasers. So kann unter Ber�cksichtigung der Flugh�he zur Bestimmung der Position ein 3D-\gls{slam}-Algortihmus eingesetzt werden. Diese Art der Lokalisierung hat gegen�ber des simplen Scanmatching-Verfahren den Vorteil, das es unabh�ngig von senkrechten Eigenschaften der Umgebung ist. Ob dies jedoch dazuf�hrt das keine L�cken in der Positionsbestimmung auftreten ist nicht garantiert. Deshalb gilt es, sich zus�tzlich mit der Entwicklung eines Notlaufprogramms zu besch�ftigen. Dieses k�nnte so aufgebaut sein, das es ausbleibende Aktualisierungen der Positionsdaten erkennt, alle von der Regelung geforderten Stellwinkel zur�ck nimmt und erst nach erhalten neuer Positionswerte wieder freigibt. Gleichzeitig w�re eine akustische Warnung denkbar, so dass der Bediener im Erstfall die Regelung komplett deaktivieren und das Flugsystem rechtzeitig manuell stabilisieren kann.

Alles in allem ist, das Ziel mit dem Quadrocopter einer autonome Refrenzplattform f�r am Fraunhofer-Insitut f�r Integrierte Schaltungen entwickelte Lokalisierungssysteme  durch diese Arbeit ein St�ck n�her ger�ckt.
